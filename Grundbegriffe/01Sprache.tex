%% =============================================================================
%% Mathematische Grundlagen - Grundbegriffe
%% Kapitel 01 - Sprache der Mathematik
%% Autor: Andreas Zeh-Marschke
%% Datum: 2025-04-21
%% =============================================================================

\chapter{Sprache der Mathematik}
\label{cha:Gdl-K01-Sprache}

Die Mathematik hat eine eigene Sprache. 

Zum einem gibt es einen Aufbau für die abstrakten Objekte, die in der 
Mathematik behandelt werden. Es ist eine Sprache, mit deren Hilfe Begriffe 
klar definiert werden und darauf aufbauend Aussagen getroffen und bewiesen 
werden. Damit sind auch bereits wichtige Begriffe genannt: 
\textbf{Definition}, \textbf{Satz} und \textbf{Beweis}. Dies sind jedoch nur 
die wichtigsten Begriffe. Weitere Begriffe werden im Abschnitt 
\ref{sec:Gdl-K01-A01} beschrieben. Alles dient dazu, ein klares und tiefes
Verständnis für die Mathematik und deren Objekte zu bekommen. 

Zum anderen ist die Mathematik geprägt durch Formeln, Zeichen und Symbole. 
Die Umgangssprache ist oftmals zu ungenau und der Quell von
Fehlinterpretationen. Daher benötigt die Mathematik einen Wortschatz und eine
Grammatik, um mathematisch schreiben zu können. Eine Sprache, die weltweit
verstanden wird. 
So wie jede fremde Sprache muss dies gelernt werden. Der Wortschatz wird nach 
und nach ausgebaut. Im Abschnitt \ref{sec:Gdl-K01-A02} werden erste Formeln 
und Symbole dargelegt. Die mathematische Sprache in Formeln und Symbolen 
versucht kurz und prägnant die mathematischen Inhalte auszudrücken, um
umgangssprachliche Ausführungen zu verdeutlichen und zu präzisieren.

%% =============================================================================
\section{Definition - Satz - Beweis}
\label{sec:Gdl-K01-A01}

%% -----------------------------------------------------------------------------
\begin{Unit}[Begriff Definition]
Eine \Begriff{Definition} führt ein neues mathematisches Objekt klar und 
eindeutig ein. Diese mathematischen Objekte sind abstrakte Objekte unseres 
Denkens. Sie kommen nicht natürlich vor, sie helfen jedoch die Natur zu 
beschreiben. Mit einer Definition wird den Lesern klar beschrieben, was unter 
einem mathematischen Objekt zu verstehen ist. Am Anfang, wenn noch keine
mathematischen Definitionen vorhanden sind, basieren die Definitionen auf
anschaulichen umgangssprachlichen Formulierungen, informelle Beschreibungen.
Teilweise stehen am Anfang auch \Begriff{Axiom}e. Ein Axiom ist ein 
grundlegender und allgemein anerkannter Grundsatz oder eine Aussage. Ein Axiom 
wird ohne Beweis akzeptiert.
\end{Unit}
\Translation{Definition}{definition}
\Translation{Axiom}{axiom}


%% -----------------------------------------------------------------------------
\begin{Unit}[Begriff Satz]
Ein \Begriff{Satz} ist eine Aussage über ein mathematisches Objekt, die 
bewiesen werden kann. Sätze werden oftmals auch als \Begriff{Theorem} (ein 
besonders wichtiger Satz) bezeichnet. Manche Theoreme und Sätze tragen manchmal 
eine eigenen Namen, wenn sie bedeutsam und auch an anderer Stelle zum Einsatz
kommen. Oftmals wird für die Vorbereitung eines Satzes andere Aussage benötigt, 
die ebenfalls bewiesen werden. Diese heißen dann \Begriff{Proposition} oder
\Begriff{Bemerkung}. Wenn diese vorbereitenden Sätze auch anderweitig eingesetzt
werden können, dann werden sie oft als \Begriff{Lemma} oder \Begriff{Hilfssatz}
bezeichnet. Manchmal kann aus einem Satz auch ohne viel Aufwand eine weitere 
Aussage gefolgert werden. Dann wird dies als \Begriff{Korollar} oder
\Begriff{Folgerung} bezeichnet. Somit gibt es verschiedene Abstufungen für Sätze.
\end{Unit}
\Translation{Satz}{theorem}
\Translation{Theorem}{theorem}
\Translation{Satz}{proposition}
\Translation{Proposition}{proposition}
\Translation{Lemma}{lemma}
\Translation{Hilfssatz}{lemma}
\Translation{Korollar}{collolary}
\Translation{Folgerung}{conclusion}

%% -----------------------------------------------------------------------------
\begin{Unit}[Begriff Beweis]
Der dritte wichtige Begriff ist der \textbf{Beweis}. Bei einem Beweis wird die
Richtigkeit einer Aussage durch logische Argumente verdeutlicht. Ein Beweis 
ist eine Ausführung, die nachvollziehbar darstellt, wieso ein Satz korrekt 
ist. Wenn etwas bewiesen ist, dann ist dieses Tatsache fix und muss nicht 
immer wieder aufs neue bewiesen werden. Nachfolgende Beweise können dann auf 
die bewiesenen Aussagen zurückgreifen. In einem späteren Kapitel (siehe 
Kapitel \ref{cha:Gdl-K03-Beweise}) werden verschiedene Arten von Beweisen
vorgestellt. Beweise können kurz sein (\enquote{Einzeiler}) oder über hunderte 
von Seiten gehen. Das Lesen einen Beweises ist nicht einfach. Das 
Nachvollziehen kann dabei sehr aufwändig sein, so dass für das Lesen einer 
Zeile auch mal einen ganzen Tag oder mehr benötigt wird, bis es 
vollständig verstanden hat. Da auch auf andere Definitionen und Sätze
zurückgegriffen wird, kann es auch notwendig sein, sich die entsprechenden
Grundlagen intensiv anzusehen.
\end{Unit}
\Translation{Beweis}{proof}

%% -----------------------------------------------------------------------------
\begin{Unit}[Weitere Begriffe]
Es gibt weitere Elemente die dazu dienen, die Begriffe und mathematischen 
Objekten genauer kennen zu lernen und zu erlernen, mit ihnen umzugehen. Dazu 
gehören insbesondere \Begriff{Beispiel}e, manchmal auch 
\Begriff{Gegenbeispiel}e, um die Definitionen zu begreifbarer zu machen.
\Begriff{Diagramm}e dienen zur grafischen Veranschaulichung von Sachverhalten.
Auch \Begriff{Berechnung}en und \Begriff{Experiment}e oder 
\Begriff{Simulation}en können das Verständnis über die Anwendung der Begriffe
vertiefen. Wichtig sind dabei auch \Begriff{Übung}en, \Begriff{Aufgabe}n und
\Begriff{Lösung}en dazu. Damit wird der Umgang mit den Begriffen geschult und
eventuell Lücken erkannt. Weitere Elemente können die Darstellung von
\Begriff{Motivation}, \Begriff{Hintergrund} oder \Begriff{Zusammenhang} sein, 
um die Begriffe und Sätze in einen weiteren Kontext einzubinden. Auch die
Darstellung der \Begriff{Historie}, wie sich Begriffe und Sätze entwickelt 
haben, kann das Verständnis fördern. Dazu zählen auch allgemeine 
\Begriff{Anmerkung}en. Im Rahmen von Berechnungen können auch ein
\Begriff{Algorithmus} oder \Begriff{Computerprogramm}e hilfreich für das 
Verständnis sein. Dies alles dient dazu, die Objekte besser zu verstehen und 
damit auch in einem anderen Kontext anwenden zu können.
\end{Unit}
\Translation{Beispiel}{example}
\Translation{Gegenbeispiel}{counterexample}
\Translation{Diagramm}{diagram}
\Translation{Berechnung}{calculation}
\Translation{Experiment}{experiment}
\Translation{Simulation}{simulation}
\Translation{Übung}{exercise}
\Translation{Übung}{practice}
\Translation{Aufgabe}{problem}
\Translation{Lösung}{solution}
\Translation{Motivation}{motivation}
\Translation{Hintergrund}{background}
\Translation{Zusammenhang}{context}
\Translation{Historie}{history}
\Translation{Anmerkung}{remark}
\Translation{Algorithmus}{algorithm}
\Translation{Computerprogramm}{computer program}

%% =============================================================================
\section{Formeln und Symbole}
\label{sec:Gdl-K01-A02}

Ein Wortschatz, das sind Zeichen und Symbole Ein Beispiel sei
\begin{align}
  \forall x \in \RR^+: \exists n_0 \in \NN : \forall n \geq n_0: 
    x > \frac{1}{n} \ .
\end{align}
Diese Formel ist die kompakte Form für die Aussage:

\enquote{Für alle positiven reellen Zahlen existiert eine natürliche Zahl 
$n_0$, so dass für alle natürlichen Zahlen $n$, die größer oder gleich $n_0$ 
sind, die Zahl 1 geteilt durch $n$ kleiner als x ist.}

Dieser Text und die obige Formel besagen das selbe. Aber auch im Text sind 
einige abstrakte Begriffe enthalten, die zuerst gelernt werden müssen. Was 
sind reelle Zahlen, was sind natürliche Zahlen. Was sind positive Zahlen? Was
bedeutet \enquote{größer oder gleich}. In der Formel sind weitere Symbole und
Zeichen enthalten. Auch diese müssen gelernt werden, deren genaue Bedeutung 
erfahren werden. Wenn der Wortschatz gelernt ist und die Grammatik, also wie 
die Zeichen und Symbole zusammengesetzt werden und welche Bedeutung dies dann 
hat, dann ist die obige Formel problemlos lesbar, egal ob die 
Muttersprache Deutsch, Englisch, Französisch, Russisch, Chinesisch oder 
welche Sprache auch immer ist.

Hierbei muss berücksichtigt werden, dass die mathematische Sprache nicht 
normiert ist. Daher kann es Abweichungen geben, manchmal nur kleine Varianten,
manchmal auch größere Abweichungen. Daher ist es immer wichtig sich 
klarzumachen, wie die Symbole und Zeichen und auch die Grammatik zu verstehen 
ist. Manchmal hängt es auch von der Umgebung ab. Die imaginäre Einheit wird in 
der Mathematik in der Regel mit $i$ bezeichnet. In der Elektrotechnik eher mit 
$j$, da das $i$ für die Stromstärke reserviert ist.

Das Kennenlernen von Wortschatz und Grammatik der mathematischen Sprache ist 
ein Ziel der nachfolgenden Kapitel. Dabei werden auch teilweise Varianten
aufgezeigt. 

Die mathematische Sprache ist nicht ein Selbstzweck für die Mathematik. Die
Mathematik ist die Sprache, in der viele Wissenschaften (insbesondere Natur-,
Ingenieur-, Wirtschaftswissenschaften, Informatik) beschrieben werden. Für 
diese Wissenschaften ist die Mathematik ein Hilfswerkzeug, ein wichtiges und
mächtiges Hilfswerkzeug, um viele Sachverhalte zu beschreiben.

%% =============================================================================
\section{Aufgaben}

\renewcommand{\mitAufgaben}{Ja}
\renewcommand{\mitLoesungen}{Nein}
%% =============================================================================
%% Mathematische Grundlagen - Grundbegriffe
%% Kapitel 01 - Sprache der Mathematik - Aufgaben und Lösungen
%% Autor: Andreas Zeh-Marschke
%% Datum: 2025-04-21
%% =============================================================================

%\ifthenelse{\equal{\mitAufgaben}{Ja}}{
%\section{Aufgaben}
%}{}
%\ifthenelse{\equal{\mitLoesungen}{Ja}}{
%\section{Lösungen zu Kapitel \ref{cha:Gdl-K01-Sprache}}
%}{}

%% -----------------------------------------------------------------------------
\ifthenelse{\equal{\mitAufgaben}{Ja}}{
\begin{Aufgabe}
Dies ist eine Aufgabe
\end{Aufgabe}
}{}

\ifthenelse{\equal{\mitLoesungen}{Ja}}{
\begin{Loesung}
Dies ist eine Lösung zu <unklar>
\end{Loesung}
}{}

%% -----------------------------------------------------------------------------
\ifthenelse{\equal{\mitAufgaben}{Ja}}{
\begin{Aufgabe}
Dies ist eine zweite Aufgabe
\end{Aufgabe}
}{}

\ifthenelse{\equal{\mitLoesungen}{Ja}}{
\begin{Loesung}
Dies ist eine zweite Lösung
\end{Loesung}
}{}




% wenn mit Aufgaben dann
%   Neuer Abschnitt "Aufgaben"
%   Datei mit Aufgaben integrieren
% In den Abschnitten ohne Lösungen, dan erst im Anhang die Lösungen integrieren
%
% notwenidieg Steuerparameter: mitAufgaben, mitLoesungeen