%% =============================================================================
%% Mathematische Grundlagen - Grundbegriffe
%% Vorwort
%% 2025-04-08 Andreas Zeh-Marschke
%% =============================================================================
\chapter*{Vorwort}

In diesem Skript werden \textbf{Mathematische Grundlagen} dargestellt. 
Dies beinhaltet die Themen 
\textbf{Sprache der Mathematik} (Kapitel \ref{cha:Gdl-K01-Sprache}),
\textbf{Aussagen und elementare Logik} (Kapitel \ref{cha:Gdl-K02-ElemLogik}),
\textbf{Beweisverfahren} (Kapitel \ref{cha:Gdl-K03-Beweise}), 
\textbf{Naive Mengenlehre} (Kapitel \ref{cha:Gdl-K04-Mengen}),
\textbf{Relationen} (Kapitel \ref{cha:Gdl-K05-Relationen}) und
\textbf{Abbildungen und Funktionen} (Kapitel \ref{cha:Gdl-K06-Abbildungen}).
Das sind somit nur die wichtigsten \textbf{Grundbegriffe}. Nicht dargestellt 
sind Zahlen, Zahlenmengen und Strukturen, die in eigenen Skripten dargestellt 
werden.

Dieses Skript entstand aus Vorlesungen, die ich an der Dualen Hochschule 
Baden-Württemberg - Karlsruhe (ehemalige Berufsakademie Karlsruhe), erstmals 
im Frühjahr 2001 im Studiengang Wirtschaftsinformatik im Fachbereich 
Wirtschaft, gehalten habe. In der Vorlesung mit dem Namen \textbf{Logik und 
Algebra} wurde das Thema Mathematische Grundlagen im Umfang von etwa 30 
Stunden inklusive Übungen behandelt - inzwischen mit etwa 24 Stunden. Daher 
können in der Vorlesung einige Themen nur angerissen werden. Für die weite 
Reise ins Innere der Mathematik und für die Frage des Einsatzes 
mathematischer Theorie in die Praxis der Anwendung kann dies nur ein 
erster Startpunkt sein.

Was soll im Rahmen eines solchen Skript behandelt werden? Was ist wichtig? 
Dies ist keine leichte Frage, denn die Mathematik und auch das Thema 
Mathematische Grundlagen ist reichhaltig und bietet viele interessante 
Aspekte, die nicht alle in ein kleines Skript zusammengefasst werden kann. 
Für die Auswahl die Frage gestellt werden, was ist für die weiteren Themen
in der Mathematik wichtig? Wichtig ist zu denken und zwar systematisch, 
logisch, abstrakt und strukturiert! Dabei darf aber der Bezug zu 
verschiedenen Anwendungen in Mathematik, Informatik, Naturwissenschaften und 
Technik nicht außer acht gelassen werden. Ich werde versuchen neben der 
Theorie auch praktische Beispiele zu bringen, um damit (hoffentlich) das 
Verständnis zu fördern.

Des weiteren möchte ich Peter Hartmann beipflichten, der im Vorwort zu 
seinem Buch \textbf{Mathematik für Informatiker} (\cite{Hartmann.2002}) 
geschrieben hat: \enquote{Genauso wie Sie eine Programmiersprache nicht durch 
das Lesen der Syntax lernen können, ist es unmöglich Mathematik zu verstehen 
ohne mit Papier und Bleistift zu arbeiten}. Das heißt, dass eine intensive
Beschäftigung mit der Mathematik notwendig ist, um die Mathematik zu 
verstehen. Dazu gehört auch viel Übung. Das ist mit Arbeit verbunden, aber 
ohne dies geht es nicht.

Anforderungen wandeln sich schnell und werden sich wohl auch immer schneller 
wandeln. Daher ist die schnelle Einarbeitung in neue Themen stets notwendig.
Dafür ist eine stabile und solide Basis nötig. Dazu ist es notwendig, dass 
man \textbf{denkt}, um das vorhandene Wissen richtig einzusetzen. Daher werde 
ich in diesem Skript auf Beweise nicht ganz verzichten, denn Beweise sind 
eine Möglichkeit, um das \emph{Denken} zu üben. Das Denken darf sich jedoch 
nicht allein auf die Mathematik konzentrieren. Wichtig ist ein übergreifendes 
Denken. Wobei dies ein langer Entwicklungsprozess sein wird, der im Rahmen nur
einer Vorlesung nicht erreicht werden kann. Dieses Skript stellt eine 
Einführung dar, um eine Basis für die Arbeit zu schaffen. Ich hoffe, dass mir 
dies einigermaßen gelungen ist, obwohl ich weiß, dass es schwierig ist, das 
mathematische Verständnis zu vermitteln.

Im Literaturverzeichnis sind einige grundlegende und weiterführende Bücher 
aufgeführt. Bücher, welche das Thema Logik und Algebra behandeln, oder 
Teilaspekte dieses Skripts beleuchten. Diese Literatur kann daher als 
Vertiefung aufgefasst werden, die jedoch in der Regel weit mehr beinhalten 
als dieses Skript. Manchmal haben die Bücher auch andere Herangehensweisen an 
die Thematik, was sehr spannend sein kann, denn es gibt viele Wege, sich die 
Mathematik zu erschließen.

Manches übernehme ich von \cite{Zeh-Marschke.2014}.
Eine schöne, sehr kompakte Ausarbeitung ist auch bei Deiser, Lasser, Vogt und 
Werner (siehe \cite{Deiser.Lasser.2016}) zu finden.

Die behandelten Inhalte werden gut von 
Hartmann \cite{Hartmann.2002}, 
Lehmann und Schulz \cite{Lehmann.2004}, 
Meinel und Mundhenk \cite{Meinel.2002}, 
Schichel und Steinbauer \cite{Schichl.2009}, 
Staab \cite{Staab.2007} und 
Struckmann und Wätjen \cite{Struckmann.2007} 
dargestellt. Diese wurden mehr oder weniger für die Erstellung des Skriptes 
zu Rate gezogen.

Die anderen Referenzen in der Literaturliste beziehen sich auf Bücher, die 
teilweise deutlich über den Stoff der Vorlesung gehen: 
Beutelspacher und Zschiegner \cite{Beutelspacher.2002}, 
Henze \cite{Henze.2005}, 
Knauer \cite{Knauer.2001}, 
Lau \cite{Lau.2004}, 
\cite{Lau.2004b}, 
Schmidt \cite{Schmidt.2000}, 
Steger und Sickinger \cite{Steger.2002}, 
Teschl und Teschl \cite{Teschl.2006} und 
Witt \cite{Witt.2001}. 
Die Bücher Arens und andere \cite{Arens.2008} und 
Eichholz und Vilkner \cite{Eichholz.2002} sind eher Nachschlagewerke.

Die erste Version des Skripts entstand 2001. Im Laufe der Jahre wurden 
Anpassungen und Ergänzungen, sowohl auf fachlicher, als auch auf technischer 
Art umgesetzt. Von daher wird das Skript ständig Veränderungen unterzogen. Das 
Skript wurde mit \TeX, genauer mit \LaTeX\ erstellt.

Ich habe versucht Fehler herauszunehmen, ohne neue Fehler zu machen - nicht 
immer ganz einfach. Wenn Fehler entdeckt werden, so bitte ich, dass mir diese 
Fehler gemeldet werden, damit ich diese Fehler in einer neuen Version 
korrigieren kann. Auch Anregungen und weitere Anmerkungen sind gerne 
willkommen. 
\vspace{1cm}


Manches übernehme ich von \cite{Zeh-Marschke.2014}.
Eine schöne, sehr kompakte Ausarbeitung ist auch von Deiser, Lasser, Vogt und 
Werner (siehe \cite{Deiser.Lasser.2016})

