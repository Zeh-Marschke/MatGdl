%% =============================================================================
%% Mathematische Grundlagen - Grundbegriffe
%% Kapitel 02 - Aussagen und elementare Logik
%% Autor: Andreas Zeh-Marschke
%% Datum: 2025-03-30
%% =============================================================================

\chapter{Aussagen und elementare Logik}
\label{cha:Gdl-K02-ElemLogik}

%\begin{unit}
In diesem Kapitel werden grundlegende Begriffe der mathematischen Logik 
eingeführt. Es werden dabei einige Themen behandelt, die für das 
mathematische Denken, die mathematische Sprechweise und das Beweisen 
benötigt werden. Es kann und soll damit im Rahmen dieser Einführung die
mathematische Logik nur insoweit eingeführt werden, wie dies für das 
weitere Verständnis benötigt wird. Als Ziel kann hier die Einführung in das 
logische Denken und die Durchführung von logischen Schlussfolgerungen genannt
werden. Logische Schlussfolgerungen, die gelernt und später auch angewendet 
werden. Ein weiteres Ziel ist es, mathematische Formulierungen kennen zu 
lernen, mathematische Formulierungen, die nicht nur in der Mathematik, 
sondern auch in anderen Bereichen (zum Beispiel Wirtschaftswissenschaften,
Ingenieurwesen, Informatik \ldots) eingesetzt werden. Die Einführung in 
diesem Kapitel ist damit nur sehr knapp und geht somit nicht bis zu den
philosophischen Fragen der Logik. Das würde den Rahmen ganz und gar 
sprengen.

Die klare Formulierung von Aussagen ist in allen Bereichen notwendig, nicht 
nur in der Mathematik. Beispielsweise ist die Spezifikation einer Software 
ohne klare Aussagen, Aussagen, die vollständig und widerspruchsfrei sind, 
nicht denkbar. Auch andere Wissenschaften erfordern klare Aussagen und 
Folgerungen aus Aussagen. Auch in Programmiersprachen sind logische Konzepte
implementiert. Um diese Konzepte zu verstehen, ist die Kenntnis von Aussagen 
eine unabdingbare Basis.

Zuerst werden \textbf{Aussagen} (Abschnitt \ref{sec:ElemLogik-Aussagen}) und 
dann \textbf{Junktoren} (Abschnitt \ref{sec:ElemLogik-Junktoren}) definiert.
Junktoren sind grundlegende Verknüpfungen zwischen Aussagen. Dann (Abschnitt
\ref{sec:ElemLogik-Verknuepfung}) werden allgemeine Verknüpfungen zwischen 
Aussagen betrachtet. Danach werden Sätze der \Begriff{Aussagenlogik} 
(Abschnitt \ref{sec:ElemLogik-Aussagenlogik}) untersucht und bewiesen. 
Im anschließenden Abschnitt \ref{sec:ElemLogik-Praedikate} wird der 
Sprachumfang der mathematischen Sprache erweitert. 
%\end{unit}

%% -----------------------------------------------------------------------------
%% Abschnitt: Aussagen und Aussageformen
%% -----------------------------------------------------------------------------
\section{Aussagen}
\label{sec:ElemLogik-Aussagen}

%% =============================================================================
%\subsection*{Aussagen}

%% -----------------------------------------------------------------------------
\begin{Unit}[Definition Aussage]
Zur präzisen mündlichen oder schriftlichen Formulierung von Sachverhalten ist 
die Mathematik, aber nicht allein die Mathematik, dazu übergegangen, Aussagen 
zu treffen, die teilweise in natürlicher Sprache und teilweise in einer 
künstlichen und formalisierten Sprache wiedergegeben werden. Was ist eine
\emph{Aussage}? Genauer gesagt, was wird unter einer Aussage verstanden, wie 
wird der Begriff definiert.

\begin{Definition}
Ein sprachliches Gebilde $A$ heißt \Begriff{Aussage}, wenn eindeutig 
entscheidbar ist, ob es wahr oder falsch ist. 
\end{Definition}
\Translation{Aussage}{proposition}

Einige Ergänzungen zur Definition:
\begin{itemize}

\item Einer Aussage kann somit eindeutig der \Begriff{Wahrheitswert}
  \enquote{\Begriff{wahr}} oder \enquote{\Begriff{falsch}} zugeordnet werden. 
  Ist $A$ eine Aussage, dann sei $w(A)$ der zugehörige Wahrheitswert der 
  Aussage.
\Translation{Wahrheitswert}{logical value}
\Translation{wahr}{true}
\Translation{falsch}{false}

\item Statt der Begriffe \enquote{wahr} und \enquote{falsch} werden oftmals 
  auch andere Begriffe oder Kurzzeichen verwendet. Es werden oftmals statt
  \enquote{wahr} auch der englische Begriff \enquote{\Begriff{true}} oder die   
  Kürzel $W$, $w$, $T$, $t$ oder $1$ verwendet. Für \enquote{falsch} wird 
  auch der englische Begriff \enquote{\Begriff{false}} oder die Kürzel $F$, 
  $f$ oder $0$ verwendet. 

\item Aussagen werden oftmals mit großen lateinischen Buchstaben bezeichnet,
  beispielsweise $A_i$ ($i \in \NN$). 

\end{itemize}

Diese Definition von Aussagen basiert auf dem Prinzip der Zweiwertigkeit, das 
heißt auf der Tatsache, dass eine Aussage entweder wahr oder falsch ist. 
Daher wird dies zweiwertiger Logik genannt. Es gibt Kulturkreise, in denen 
das Prinzip der zweiwertigen Logik nicht gilt, in denen es neben \enquote{wahr}
\ und \enquote{falsch} auch noch andere Wahrheitswerte gibt, zum Beispiel
\enquote{vielleicht} oder \enquote{unbestimmt} oder sogar noch andere 
Abstufungen.\footnote{siehe hierzu John D. Barrow; Ein Himmel voller Zahlen 
- Auf den Spuren mathematischer Wahrheit; rororo, 1999} 
\end{Unit}

%% -----------------------------------------------------------------------------
\begin{Unit}[Beispiele]
Dazu einige Beispiele, um den Begriff \emph{Aussage} genauer zu fassen. 
\begin{enumerate}

\item \textbf{2 + 3 = 5.} \\
Dies ist eine wahre Aussage.

\item \textbf{4 ist eine Primzahl.} \\
Dies ist eine falsche Aussage. Das bedeutet, dass Aussagen nicht immer wahr 
sind, sie können auch falsch sein!

\item \textbf{Es gibt unendlich viele Primzahlen.} \\
Dies ist eine wahre Aussage, die bereits vom griechischen Mathematiker 
Euklid\index[names]{Euklid}\footnote{Euklid von Alexandria (um 300 v.Chr.),
griechischer (\url{https://de.wikipedia.org/wiki/Euklid})} vor etwa 2.300 
Jahren bewiesen wurde.

\item \textbf{Die Gleichung $x^n + y^n = z^n$ hat, außer der trivialen 
Lösung, keine ganzzahligen Lösungen für $n$ größer als 2} \\
Dies ist eine Aussage, die Pierre de Fermat\index[names]{Fermat, Pierre de}
\footnote{Pierre de Fermat (1601 - 1665), französischen Mathematiker 
(\url{https://de.wikipedia.org/wiki/Pierre_de_Fermat})\\
Eine Kurzbiographie über Pierre de Fermat: Klaus Barner; Das Leben Fermats; 
in Mitteilungen der Deutschen Mathematiker-Vereinigung, Heft 3/2001}
etwa 1637 aufgestellt hat. Die Aussage ist als großer Fermat'scher Satz oder 
Fermats letzter Satz bekannt. Sie wurde erst 1993 / 1995 von Andrew Wiles
\index[names]{Wiles, Andrew} 
\footnote{Andrew Wiles (*1953), britischer Mathematiker, 
(\url{https://de.wikipedia.org/wiki/Andrew_Wiles})} bewiesen\footnote{Eine
interessante und lesenswerte Beschreibung der Geschichte des Satzes von 
Fermat, von den Grundlagen bis zu seiner Lösung, mit vielen historischen 
Anmerkungen steht in Simon Singh; Fermats letzter Satz; dtv, 2000}.

\item \textbf{Jede gerade Zahl, die größer als 2 ist, ist Summe zweier 
Primzahlen.}\\
Die Aussage ist die Goldbach'sche Vermutung.\footnote{ 
  \url{https://de.wikipedia.org/wiki/Goldbachsche_Vermutung}, benannt nach
  Christian Goldbach\index[names]{Goldbach, Christian} (1690-1764),
  deutscher Mathematiker,
  (\url{https://de.wikipedia.org/wiki/Christian_Goldbach)}} 
(\emph{Beispiele}: 4 = 2 + 2, 6 =  3 + 3, 8 = 3 + 5, 10 = 5 + 5 = 3 + 7) 
Dies ist eine Aussage, von der noch nicht bekannt ist, ob sie wahr oder 
falsch ist.

\item \textbf{Es ist Vollmond.} \\
Dies ist eine Aussage, die manchmal wahr ist, manchmal jedoch falsch. Der
Wahrheitsgehalt hängt von der Zeit ab. Dies wird in der \textbf{temporalen 
Logik}\index{temporale Logik} 
(\url{https://de.wikipedia.org/wiki/Temporale_Logik}) betrachtet. Dort werden
zeitliche Zusammenhänge mit einbezieht.
\Translation{Temporale Logik}{temporal logic}

\item \textbf{Guten Morgen!}\\
Dies ist keine Aussage, sondern ein Ausruf.
\end{enumerate}
\end{Unit}

%% -----------------------------------------------------------------------------
\begin{Unit}[Diskussion]
Nun wird der Satz \enquote{\emph{Dieser Satz ist falsch.}} untersucht. Auf 
den ersten Blick ist es eine Aussage. Wenn es eine Aussage ist, dann ist zu 
klären, welchen Wahrheitswert die Aussage hat, denn einer Aussage muss nach 
der Definition eindeutig ein Wahrheitswert zugeordnet werden können. 

Wenn es eine wahre Aussage ist, dann besagt der Satz, dass die Aussage des 
Satzes falsch ist, dass es also eine falsche Aussage ist. Also kann es keine 
wahre Aussage sein. Ist es dann eine falsche Aussage? Wenn die Aussage des 
Satzes falsch ist, dann besagt dies, dass die Aussage des Satzes wahr ist!? 
Auch das führt zu einem Widerspruch. Es kann nicht entschieden werden, ob der 
Satz wahr oder falsch ist. Somit ist es keine Aussage (nach der Definition)! 
Es ist eine Paradoxie, die zu tiefer gehenden, philosophischen Problemen der 
Logik führt, die hier nicht näher beleuchtet werden.
\end{Unit}

%% -----------------------------------------------------------------------------
\begin{Unit}[Diskussion] 
Beim Satz \enquote{\emph{Diese Person ist groß.}} ist es ebenfalls schwer zu
bestimmen, ob dieser Satz wahr oder falsch ist. Die Größe bezieht sich hierbei 
auf die Körpergröße. Für Pygmäen ist eine Person mit 1,70 m Körperlänge eine 
große Person, für Basketballspieler ist dies jedoch nicht groß. Dies führt in 
die moderne Entwicklung der \Begriff{Fuzzylogik} 
(\url{https://de.wikipedia.org/wiki/Fuzzylogik}), die bewusst mit Unschärfen
arbeitet. Die Fuzzylogik wird hier nicht weiterverfolgt. Die Fuzzylogik wird
beispielsweise in der Regelung für die Steuerung in Technik aber auch in der 
Medizin eingesetzt.
\end{Unit}
\Translation{Fuzzylogik}{fuzzy logic}

%%------------------------------------------------------------------------------
%% Abschnitt: Verknüpfung von Aussagen
%%------------------------------------------------------------------------------
%%
\section{Junktoren}
\label{sec:ElemLogik-Junktoren}

\begin{Unit}[Anmerkung]
Im Nachfolgenden werden grundlegende Verknüpfungen von Aussagen betrachtet. 
Diese werden \Begriff{Junktoren} genannt. Damit werden Aussagen verknüpft. 
Der Wahrheitswert hängt dabei von den Wahrheitswerten der verknüpften 
Aussagen und der Art der Verknüpfung ab. Auch wenn die Verknüpfungen 
teilweise aus dem Alltagsleben bekannt sind, so ist deren mathematische 
Definition und Deutung manchmal etwas anders als in der Umgangssprache. Es 
sind dies die Verknüpfungen \enquote{nicht}, \enquote{und}, \enquote{oder},
\enquote{wenn, dann}\ und \enquote{genau dann, wenn}. In den Definitionen 
der Verknüpfungen werden diese Verknüpfungen so festgelegt, dass sich der
Wahrheitswert der Verknüpfung \textbf{eindeutig} aus den Wahrheitswerten der
Teilaussagen ergibt, unabhängig davon, ob zwischen den Teilaussagen ein 
inhaltlicher Zusammenhang besteht oder jedoch kein Zusammenhang besteht. 
\end{Unit}
\Translation{Junktor}{logical connective}

%% -----------------------------------------------------------------------------
\begin{Unit}[Definition Wahrheitstafel]

Zur Darstellung der Aussagen wird eine \Begriff{Wahrheitstafel} oder
\Begriff{Verknüpfungstafel} verwendet. In der Tabelle sind für alle möglichen
Belegungen der Variablen mit Wahrheitswerten der daraus resultierende 
Wahrheitswert der Verknüpfung dargestellt. Die Wahrheitstafel in Tabelle
\ref{tbl:Beispiel Wahrheitstafel} zeigt als Beispiel die Wahrheitstafel für 
die Junktoren, die im Nachfolgenden dann genauer beschrieben und erläutert 
werden.
\Translation{Wahrheitstafel}{truth table}

\begin{table}[htbp] 
\begin{center}
  \begin{tabular}{c|c||c|c|c|c|c}
    $A$ & $B$ & $\neg A$ & $A \land B$ & $A \lor B$ & $A \rightarrow B$ 
      & $A \leftrightarrow B$\\ \hline
    $w$ & $w$ & $f$ & $w$ & $w$ & $w$ & $w$ \\
    $w$ & $f$ & $f$ & $f$ & $w$ & $f$ & $w$ \\
    $f$ & $w$ & $w$ & $f$ & $w$ & $w$ & $w$ \\
    $f$ & $f$ & $w$ & $f$ & $f$ & $w$ & $w$ \\
  \end{tabular}
  \caption{Wahrheitstafel}
  \label{tbl:Beispiel Wahrheitstafel}
\end{center} 
\end{table}
\end{Unit}

%% =============================================================================
\subsection*{Negation}

%% -----------------------------------------------------------------------------
\begin{Unit}[Definition Negation] Zuerst wird die \enquote{nicht}-Verknüpfung
präzisiert.

\begin{Definition}
Es sei $A$ eine beliebige Aussage. Eine Aussage $C$ heißt \Begriff{Negation} 
der Aussage $A$ oder \Begriff{nicht} $A$, falls $C$ genau dann wahr ist, 
wenn $A$ falsch ist und falsch, wenn $A$ wahr ist. 
\end{Definition}
\Translation{Negation}{negation}
\Translation{nicht}{not}

Die Negation von $A$ wird auch mit $C = \neg A$ oder mit $C = not(A)$ oder 
mit $C = \overline{A}$ bezeichnet. Die Negation von $A$ wird in der
Wahrheitstafel in Tabelle \ref{tbl:Beispiel Wahrheitstafel} dargestellt.

Auch wenn hier nur eine Aussage verwendet wird, wird hier von einer 
Verknüpfung gesprochen, einer 1-stelligen Verknüpfung. Diese Verknüpfung 
entspricht auch dem umgangssprachlichen Umgang \enquote{nicht}.
\end{Unit}

%% -----------------------------------------------------------------------------
\begin{Unit}[Beispiel]
Ist $A$ die Aussage, dass eine Kugel rot ist, dann ist $\neg A$ die Aussage, 
dass die Kugel nicht rot ist.
\end{Unit}

%% =============================================================================
\subsection*{Konjunktion}

%% -----------------------------------------------------------------------------
\begin{Unit}[Konjunktion] Jetzt kommt die \enquote{und}-Verknüpfung.

\begin{Definition}
Es seien $A$ und $B$ beliebige Aussagen. Eine Aussage $C$ heißt
\Begriff{Konjunktion} oder \textbf{$und$-Verknüpfung}\index{und-Verknüpfung}
\index{Verknüpfung, und-} der Aussagen $A$ und $B$, falls $C$ genau dann wahr 
ist, wenn $A$ wahr ist \textbf{und} $B$ wahr ist. 
\end{Definition}
\Translation{Konjunktion}{logical conjunction}
\Translation{und}{and}

Die Konjunktion von $A$ und $B$ wird mit $C = A \land B$ oder mit 
$C = A\ and \ B$ bezeichnet. 

Die Konjunktion von $A$ und $B$ ist in der Wahrheitstafel in Tabelle
\ref{tbl:Beispiel Wahrheitstafel} dargestellt. Die Konjunktion entspricht 
dem umgangssprachlichen \enquote{und}.

Alle möglichen Kombinationen der Wahrheitswerte für $A$ und $B$ sind mit dem
Ergebnis der Verknüpfung dargestellt.
\end{Unit}

%% -----------------------------------------------------------------------------
\begin{Unit}[Beispiel] \ 

(1) Ist $A$ die Aussage, dass eine Kugel rot ist und $B$ die Aussage, dass 
auf einer Kugel eine gerade Zahl ist, dann ist die Aussage $C = A \land B$ 
die Aussage, dass eine Kugel rot ist und eine gerade Nummer hat.

(2) Ist $A$ die wahre Aussage, \enquote{2 ist gerade} und $B$ die falsche 
Aussage \enquote{3 ist gerade}, so ist die Aussage $A \land B$ 
(\enquote{2 ist gerade und 3 ist gerade}) eine falsche Aussage.
\end{Unit}

%% =============================================================================
\subsection*{Disjunktion}

%% -----------------------------------------------------------------------------
\begin{Unit}[Definition Disjunktion] Jetzt kommt die 
\enquote{oder}-Verknüpfung.

\begin{Definition}
Es seien $A$ und $B$ beliebige Aussagen. Eine Aussage $C$ heißt
\Begriff{Disjunktion} oder \Begriff{Adjunktion} oder 
\textbf{$oder$-Verknüpfung}\index{oder-Verknüpfung}\index{Verknüpfung, oder-} 
oder \textbf{$or$-Verknüpfung}\index{or} der Aussagen $A$ und $B$, falls $C$ 
genau dann wahr ist, wenn $A$ wahr ist oder $B$ wahr ist (oder $A$ wahr ist 
und $B$ wahr ist). 
\end{Definition}
\Translation{Disjunktion}{logical disjunktion}
\Translation{oder}{or}

Die Disjunktion von $A$ und $B$ wird mit $C = A \lor  B$ oder $C = A\ or\ B$ 
bezeichnet.

Die Verwendung von \enquote{$oder$} im mathematischen Sinne ist ein nicht
ausschließendes \enquote{oder}. In der Umgangssprache wird oftmals das
\enquote{oder} oftmals als \enquote{exklusives oder}, also als 
\enquote{entweder \ldots oder}, verwendet. Dies ist zu beachten. Hier wird 
mit der $oder$-Verknüpfung stets das nicht-ausschließende \enquote{oder} 
verwendet. Ein \enquote{ausschließendes oder} wird später noch eingeführt.

Die Disjunktion von $A$ und $B$ wird in der Wahrheitstafel in Tabelle
\ref{tbl:Beispiel Wahrheitstafel} dargestellt.
\end{Unit}

%% -----------------------------------------------------------------------------
\begin{Unit}[Beispiel] \ 

(1) Ist $A$ die Aussage, dass eine Kugel rot ist und $B$ die Aussage, dass 
auf einer Kugel eine gerade Zahl ist, dann ist die Aussage $C = A \lor B$ die
Aussage, dass die Kugel rot ist oder eine gerade Nummer hat. Die Kugel kann 
auch rot sein und eine gerade Nummer haben.

(2) Ist $A$ die wahre Aussage, \enquote{2 ist gerade} \ und $B$ die falsche 
Aussage \enquote{3 ist gerade}, so ist die Aussage $A \lor B$ (\enquote{2 ist 
gerade oder 3 ist gerade}) eine wahre Aussage.
\end{Unit}

%% =============================================================================
\subsection*{Implikation}

%% -----------------------------------------------------------------------------

\begin{Unit}[Definition Implikation] Bei der \enquote{wenn-dann}-Verknüpfung 
wird der Vergleich zum umgangssprachlichen noch etwas schwieriger.

\begin{Definition}
Es seien $A$ und $B$ beliebige Aussagen. Eine Aussage $C$ heißt
\Begriff{Implikation} oder \Begriff{Subjunktion} oder \Begriff{Folgerung} von 
$A$ nach $B$, falls $C$ genau dann wahr ist, wenn $A$ und $B$ wahr sind oder 
aber $A$ falsch ist. Die Implikation von $A$ nach $B$ wird mit 
$C = A \rightarrow B$ bezeichnet. Es heißt dann auch \enquote{aus $A$
\Begriff{folgt} $B$}.
\end{Definition}
\Translation{Implikation}{material conditional}
\Translation{folgt}{implies}

Die Implikation von $A$ und $B$ wird in der Wahrheitstafel in der Tabelle
\ref{tbl:Beispiel Wahrheitstafel} dargestellt.

Mit dieser Definition haben viele Personen, die vom normalen Sprachgebrauch 
ausgehen Schwierigkeiten, denn in der vorletzten Zeile steht, dass von etwas
Falschem etwas Richtiges gefolgert werden kann. Das ist auf den ersten Blick
merkwürdig, aber es wurde oben bereits erwähnt, dass die mathematische 
Definition manchmal etwas anders ist. Ein einfaches Beispiel ist in 
Beispiel \ref{bsp:ImplikationUnwahr} zu finden.
\end{Unit}


%% -----------------------------------------------------------------------------
\begin{Unit}[Beispiel]
Es sei $A$ die wahre Aussage \enquote{2 ist gerade}\ und $B$ die wahre 
Aussage \enquote{4 ist gerade}. Die Aussage $A \rightarrow B$ ist ebenfalls 
wahr, da beide Teilaussagen wahr sind. 
\end{Unit}

%% -----------------------------------------------------------------------------
\begin{Unit}[Beispiel]
\label{bsp:ImplikationUnwahr}
Es sei $A$ die falsche Aussage \enquote{1 = -1} und $B$ die wahre Aussage
\enquote{$1^2 = (-1)^2$}. Auch die Aussage $A \rightarrow B$ ist wahr.

Aus etwas Falschem kann alles gefolgert werden! Das ist das, was vielen 
Probleme macht. Aber aus einer falschen Aussage kann alles gefolgert werden, 
etwas Richtiges oder etwas Falsches. Aus der Sicht der Logik sind dann diese
entsprechenden Folgerungen wahr, nicht die Teilaussage $B$, sondern die 
komplette Folgerung.

Daher ist es wichtig, dass bei der Bearbeitung die klare mathematische 
Definition beachtet wird und nicht das, was umgangssprachlich darunter 
verstanden wird.
\end{Unit}

%% =============================================================================
\subsection*{Äquivalenz}

%% -----------------------------------------------------------------------------
\begin{Unit}[Definition Äquivalenz] Nun wird die 
\enquote{genau dann - wenn}-Verknüpfung vorgestellt.

\begin{Definition}
Eine Aussage $C$ heißt \Begriff{Äquivalenz} oder \Begriff{Bijunktion} von $A$ 
und $B$, falls $C$ genau dann wahr ist, wenn $A$ wahr ist und $B$ wahr ist 
oder aber $A$ falsch ist und $B$ falsch ist. Die Äquivalenz von $A$ und $B$ 
wird mit $C = A \leftrightarrow B$ bezeichnet.
\end{Definition}
\Translation{Äquivalenz}{logical biconditional}
\Translation{Äquivalenz}{equivalence}

Die Äquivalenz von $A$ und $B$ wird in der Wahrheitstafel in Tabelle
\ref{tbl:Beispiel Wahrheitstafel} dargestellt. Die beiden Aussagen $A$ und 
$B$ haben den selben Wahrheitswert. Beide sind wahr oder beide sind falsch. 
Die Aussage $A$ ist genau dann wahr, wenn auch Aussage $B$ wahr ist.

Zwei Aussagen heißen \Begriff{logisch äquivalent}, wenn sie für jede 
Belegung der Teilaussagen jeweils denselben Wahrheitswert erzeugen. 
\end{Unit}

%% -----------------------------------------------------------------------------
\begin{Unit}[Beispiel]
Es sei $A$ die Aussage, dass der Schalter (im Stromkreis) geschlossen ist und 
$B$ die Aussage, dass der Strom im Stromkreis fließt. Dann sind die Aussagen 
$A$ und $B$ äquivalent.
\end{Unit}

%% -----------------------------------------------------------------------------
\begin{Unit}[Anmerkung]
Damit sind die fünf wichtigsten Verknüpfungen (Negation, Konjunktion, 
Disjunktion, Implikation und Äquivalenz) eingeführt. Sie werden auch
\Begriff{Junktoren} genannt.
\Translation{Junktor}{sentential connective}

Bei der Darstellung von Ausdrücken kann es zu Problemen kommen, wenn keine 
Klammern gesetzt werden oder wenn keine Vereinbarung über die Reihenfolge 
der Verknüpfungen festlegt ist. Beispiel: $A \land B \lor C$.

Es wird daher vereinbart: Die Junktoren $\neg$, $\land$, $\lor$, 
$\rightarrow$ und $\leftrightarrow$ binden jeweils stärker als die 
Nachfolger in dieser Liste. Daher kann der Ausdruck $A \land B \lor C$ 
auch als $(A \land B) \lor C$ geschrieben werden, während 
$A \land (B \lor C)$ etwas anderes darstellt.

Im Zweifelsfall sollten eher etwas mehr Klammern geschrieben werden als 
zu wenige, um mögliche Fehlerquellen oder genauer Interpretationsfehler 
zu vermeiden.
\end{Unit}

%%------------------------------------------------------------------------------
%% Abschnitt: Verknüpfung von Aussagen
%%------------------------------------------------------------------------------
%%
\section{Verknüpfung von Aussagen}
\label{sec:ElemLogik-Verknuepfung}

%% =============================================================================
\subsection*{Verknüpfungen}

%% -----------------------------------------------------------------------------
\begin{Unit}[Anmerkung]
Ausgehend von elementaren Aussagen  können mit Hilfe der Junktoren weitere 
Aussagen, zusammengesetzte Aussagen, aufgebaut werden. Sind $A_1$ und $A_2$ 
zwei Aussagen, dann sind auch $(\neg A_1)$, $(A_1 \land A_2)$, 
$(A_1 \lor A_2)$, $(A_1 \rightarrow A_2)$ und $(A_1 \leftrightarrow A_2)$ 
wieder Aussagen. Damit können dann immer umfangreichere Aussagen gebildet 
werden. Diese zusammengesetzten Aussagen heißen auch 
\Begriff{aussagenlogische Formeln}. Wenn der Zusammenhang klar ist auch 
kurz \textbf{Formeln}.
\Translation{aussagenlogische Formel}{aussagenlogische Formel}

Wenn es bezüglich der Klammern keine Missverständnisse gibt, können Klammern
weggelassen werden. Hiermit werden Ausgaben formal aufgebaut. Die Syntax 
beschriebt den Aufbau der Aussagen. Die Bedeutung, die Semantik, hängt von 
den Belegungen der elementaren Aussagen ab. Die Bedeutung einer 
umfangreichen Bedeutung kann dann Stück für Stück berechnet werden.

Auf die Syntax und Semantik wird hier nicht genauer eingegangen. Dies kann
beispielsweise bei Witt \cite{Witt.2001} nachgesehen werden. 
\end{Unit}

%% -----------------------------------------------------------------------------
\begin{Unit}[Beispiel]
Mit Hilfe einer Wahrheitstafel können auch zusammengesetzte Aussagen 
schrittweise gelöst werden. Dazu wird die zusammengesetzte Aussage 
$(A \rightarrow B) \lor (A \land C)$ betrachtet. Wie sieht der Wahrheitswert 
der Aussage in Abhängigkeit von den Wahrheitswerten der elementaren Aussagen 
$A$, $B$ und $C$ aus. (siehe Tabelle 
\ref{tbl:Beispiel Auswertung Wahrheitstafel})

\begin{table}[htbp] 
\begin{center}
  \begin{tabular}{c|c|c||c|c|c|c|c|c|c|}
    $A$ & $B$ & $C$ & $(A$ & $\rightarrow$ & $B)$ & $\lor$ & $(A$ & $\land$ 
      & $C)$ \\ \hline
    $w$ & $w$ & $w$ & $w$ & $w$ & $w$ & $w$ & $w$ & $w$ & $w$ \\
    $w$ & $w$ & $f$ & $w$ & $w$ & $w$ & $w$ & $w$ & $f$ & $f$ \\
    $w$ & $f$ & $w$ & $w$ & $f$ & $f$ & $w$ & $w$ & $w$ & $w$ \\
    $w$ & $f$ & $f$ & $w$ & $f$ & $f$ & $f$ & $w$ & $f$ & $f$ \\
    $f$ & $w$ & $w$ & $f$ & $w$ & $w$ & $w$ & $f$ & $f$ & $w$ \\
    $f$ & $w$ & $f$ & $f$ & $w$ & $w$ & $w$ & $f$ & $f$ & $f$ \\
    $f$ & $f$ & $w$ & $f$ & $w$ & $f$ & $w$ & $f$ & $f$ & $w$ \\
    $f$ & $f$ & $f$ & $f$ & $w$ & $f$ & $w$ & $f$ & $f$ & $f$ \\ \hline
      &   &   & 0.& 1.& 0.& 2.& 0.& 1.& 0.\\
  \end{tabular}
  \caption{Beispiel Auswertung Wahrheitstafel}
  \label{tbl:Beispiel Auswertung Wahrheitstafel}
\end{center} 
\end{table}
\end{Unit}

%% -----------------------------------------------------------------------------
\begin{Unit}[Anmerkung]
Die zusammengesetzte Aussage wurde schrittweise gelöst. In der untersten 
Zeile steht, in welcher Reihenfolge die Teilaussagen gelöst wurden. Dabei 
steht \enquote{0.} für die einfache Übertragung der Basiswerte. Gleiche 
Zahlenwerte in der letzten Zeile bedeuten, dass die Ergebnisse auf dieser 
Ebene in beliebiger Reihenfolge gebildet werden können.

Die Erstellung von Wahrheitstafeln für Aussagen kann auch mit Hilfe eines
Tabellenkalkulationsprogramms erstellt werden. Diese Programme bieten 
logische Funktionen und Ausdrücke. Auch moderne Programmiersprachen bieten
Konstrukte für die Bearbeitung von logischen Ausdrücken.
\end{Unit}

%% -----------------------------------------------------------------------------
\begin{Unit}[Definition logisch Äquivalent]
Zwei zusammengesetzt Aussagen können unterschiedlich aufgebaut sein
(unterschiedliche Syntax) jedoch die selbe Bedeutung (gleiche Semantik) 
haben.

\begin{Definition}
Zwei Aussagen $A_1$ und $A_2$ heißen \Begriff{logisch Äquivalent} 
$A_1 \equiv A_2$, wenn Sie für jede Belegung der Aussagen mit 
Wahrheitswerten den selben Wahrheitswert haben.
\end{Definition}
\Translation{logisch äquivalent}{logical equivalent}

Statt $\equiv$ wird auch manchmal $\Leftrightarrow$ geschrieben.
\end{Unit}

%% =============================================================================
\subsection*{1-stellige Verknüpfungen}

%% -----------------------------------------------------------------------------
\begin{Unit}[Definition 1-stellige Verknüpfungen]
Es sei $A$ eine Aussage, dann gibt es vier 1-stelligen Verknüpfungen. Das 
heißt, es gibt vier Möglichkeiten, wie die Wahrheitswerte in Abhängigkeit 
des Wahrheitswertes von $A$ verteilt sein können. In der Tabelle 
\ref{tbl:Liste 1-stellige Verknüpfungen} werden die möglichen 
1-stelligen Verknüpfungen dargestellt:

\begin{table}[htbp]
\begin{center}
  \begin{tabular}{c||c|c|c|c}
      & $V_1$ & $V_2$ & $V_3$  & $V_4$ \\
    $A$ & $true$  & $id(A)$ & $\neg A$ & $false$ \\  \hline
    $w$ & $w$     & $w$     & $f$      & $f$     \\
    $f$ & $w$     & $f$     & $w$      & $f$     \\
  \end{tabular}
  \caption{Liste 1-stellige Verknüpfungen}
  \label{tbl:Liste 1-stellige Verknüpfungen}
\end{center} 
\end{table}

In dieser Wahrheitstafel sind in der ersten beiden Zeilen die Aussagen oder
einstelligen Verknüpfungen aufgeführt, in den beiden nachfolgenden Zeilen 
die möglichen Verteilungen der Wahrheitswerte von $V_i(A)$, für 
$i = 1, 2, 3, 4$ bei gegebenem Wahrheitswert der Aussage $A$. In der zweiten 
Zeile stehen die kurzen Bezeichnungen der Verknüpfung. Die Bezeichnungen 
dieser Verknüpfungen sind in der Literatur allerdings nicht einheitlich.

Die Verknüpfungen $\Begriff{true}(A)$ und $\Begriff{false}(A)$ sind 
unabhängig vom Wahrheitswert von $A$ stets $wahr$ beziehungsweise $falsch$. 
Die Verknüpfung $\Begriff{id}(A)$ ist die \Begriff{Identität} von $A$, so 
dass als einzige nicht-triviale 1-stellige Verknüpfung die \Begriff{Negation} 
$\neg A$ bleibt.
\end{Unit}
\Translation{Idendität}{identity}
\Translation{Negation}{negation}

%% =============================================================================
\subsection*{2-stellige Verknüpfungen}

%% -----------------------------------------------------------------------------
\begin{Unit}[Definition 2-stellige Verknüpfungen]
Nun werden die 2-stelligen Verknüpfungen untersucht. Einige dieser 
Verknüpfungen haben besondere Namen und sind (insbesondere in der Informatik)
bedeutend. Vier Verknüpfungen wurden oben bereits definiert: Konjunktion,
Disjunktion, Implikation und Äquivalenz. Jetzt werden alle Möglichkeiten 
untersucht.

Zweistellige Verknüpfungen gibt es insgesamt 16, die in den Tabellen 
\ref{tbl:Liste 2-stellige Verknüpfungen} vollständig aufgeführt sind. In der 
Tabelle stehen die Wahrheitswerte von $V_i(A,B)$ für $i = 1, 2, \ldots, 16$ in
Abhängigkeit der Wahrheitswerte von $A$ und $B$. \vspace{1ex}

\begin{table}[htbp] 
\begin{center}
\begin{tabular}{c|c||c|c|c|c|c|c|c|c}
 & & \hspace*{1.00cm} & \hspace*{1.00cm} & \hspace*{1.00cm} &   
  \hspace*{1.00cm} & \hspace*{1.00cm} & \hspace*{1.00cm} & 
  \hspace*{1.00cm} & \hspace*{1.00cm} \\
 & & $V_1$ & $V_2$ & $V_3$ & $V_4$ & $V_5$ & $V_6$ & $V_7$ & $V_8$ \\
 $A$ & $B$ & $true$ & $\lor$ &  & $id(A)$ & $\rightarrow$ 
   & $id(B)$ & $\leftrightarrow$ & $\land$ \\ \hline
  $w$ & $w$ & $w$ & $w$ & $w$ & $w$ & $w$ & $w$ & $w$ & $w$ \\
  $w$ & $f$ & $w$ & $w$ & $w$ & $w$ & $f$ & $f$ & $f$ & $f$ \\
  $f$ & $w$ & $w$ & $w$ & $f$ & $f$ & $w$ & $w$ & $f$ & $f$ \\
  $f$ & $f$ & $w$ & $f$ & $w$ & $f$ & $w$ & $f$ & $w$ & $f$ \\
\end{tabular}

\vspace*{1.0ex}

\begin{tabular}{c|c||c|c|c|c|c|c|c|c}
 & & \hspace*{1.00cm} & \hspace*{1.00cm} & \hspace*{1.00cm} & 
  \hspace*{1.00cm} & \hspace*{1.00cm} & \hspace*{1.00cm} & 
  \hspace*{1.00cm} & \hspace*{1.00cm} \\
 & & $ V_9$ & $V_{10}$ & $V_{11}$ & $V_{12}$ & $V_{13}$ & $V_{14}$ & 
   $V_{15}$ & $V_{16}$ \\
$A$ & $B$ & $\lnand$ & $\lxor$ & $\neg B$ & & $\neg A$ & & $\lnor$ & $false$ \\
  \hline
  $w$  & $w$  & $f$  & $f$  & $f$  & $f$  & $f$  & $f$  & $f$  & $f$ \\
  $w$  & $f$  & $w$  & $w$  & $w$  & $w$  & $f$  & $f$  & $f$  & $f$ \\
  $f$  & $w$  & $w$  & $w$  & $f$  & $f$  & $w$  & $w$  & $f$  & $f$ \\
  $f$  & $f$  & $w$  & $f$  & $w$  & $f$  & $w$  & $f$  & $w$  & $f$ \\
\end{tabular}
  \caption{Liste 2-stellige Verknüpfungen}
  \label{tbl:Liste 2-stellige Verknüpfungen}
\end{center} 
\end{table}

Aus der Wahrheitstafel ist zu erkennen, dass die Verknüpfungen aus dem 
zweiten Teil aus der Negation der Verknüpfungen aus dem ersten Teil 
entstehen. Es gilt $V_i(A,B) = \neg(V_{17-i}(A,B))$ für 
$i = 1, \ldots, 16$.
\end{Unit}

%% -----------------------------------------------------------------------------
\begin{Unit}[Bemerkung]
Die Verknüpfung $V_3$ setzt sich zusammen aus $A \lor \neg B$ oder aber 
$B \rightarrow A$. Die Verknüpfungen $V_1$, $V_4$, $V_6$ und $V_{16}$ sind 
trivial, so dass sich alle 2-stelligen Verknüpfungen auf die fünf 
Verknüpfungen $\neg$, $\lor$, $\land$, $\rightarrow$ und $\leftrightarrow$
zurückführen lassen. Daraus ergibt sich die

\begin{Bemerkung}
Die 2-stelligen Verknüpfungen lassen sich alle allein mit Hilfe der 
Junktoren darstellen.
\end{Bemerkung}
\end{Unit}

%% -----------------------------------------------------------------------------
\begin{Unit}[Bemerkung]
\label{bem:Aus-FolgerungVereinfachen}
Darüber hinaus lassen sich die Verknüpfungen \enquote{$\rightarrow$}\ und 
\enquote{$\leftrightarrow$} allein mit Hilfe der Verknüpfungen 
\enquote{$\neg$}, \enquote{$\lor$} und \enquote{$\land$} darstellen, denn es 
gilt: 

\begin{Bemerkung} 
Es seien $A$ und $B$ beliebige Aussagen dann gelten
\begin{align}
  (A \leftrightarrow B) \equiv (A \rightarrow B) \land 
  (B \rightarrow A) \\
  (A \rightarrow B) \equiv \neg A \lor B \quad .
\end{align}
\end{Bemerkung}

Dies kann leicht mit Hilfe der Wahrheitstafel bewiesen werden.

Die zusammengesetzten Aussagen $A \rightarrow B$ und $\neg A \lor B$ sind
syntaktisch, also formal, unterschiedlichen, jedoch semantisch, von der 
Bedeutung her, identisch, also logisch äquivalent.
\end{Unit}

%% -----------------------------------------------------------------------------
\begin{Unit}[Bemerkung]
Damit ergibt sich sogar die

\begin{Bemerkung}
Die 2-stelligen Verknüpfungen lassen sich alle allein mit Hilfe der 
Verknüpfungen \enquote{$\neg$}, \enquote{$\lor$} und \enquote{$\land$} 
darstellen.
\end{Bemerkung}

Dies ist auch der Grund, wieso in einigen Programmiersprachen, zum Beispiel 
in \emph{Java} oder in \emph{Python}, und in Tabellenkalkulationsprogrammen 
oftmals nur die logischen Operatoren \enquote{$\neg$}, \enquote{$\lor$} und
\enquote{$\land$} realisiert sind.
\end{Unit}

%% -----------------------------------------------------------------------------
\begin{Unit}[Definition $nand$, $nor$, $xor$]
Einige Verknüpfungen aus der obigen Tabelle 
\ref{tbl:Liste 2-stellige Verknüpfungen} haben einen besonderen Namen, die 
in der Tabelle auch bereits eingetragen sind.

\begin{itemize}

\item \Begriff{$\lnand$} (nicht und; nicht zugleich; 
\textbf{$nand$}\index{nand}), \\
  $A \lnand B \equiv \neg (A \land B)$, 

\item \Begriff{$\lnor$} (nicht oder; weder noch; \textbf{$nor$}\index{nor}), 
  \\
  $A \lnor B \equiv \neg (A \lor B)$, 
    
\item \Begriff{$\lxor$} (exclusive or; exklusives oder; entweder oder; 
  \textbf{$xor$}\index{xor}), \\
  $A \lxor B \equiv \neg (A \leftrightarrow B)$ .
\end{itemize}

Das \enquote{exklusiv oder} ist oftmals das oder, das umgangssprachlich 
verwendet wird.
\end{Unit}

%% -----------------------------------------------------------------------------
\begin{Unit}[Bemerkung]
Die Verknüpfungen haben ihre besondere Bedeutung in der Schaltalgebra, die 
auf der nachfolgenden Bemerkung beruht.

\begin{Bemerkung}
Die 2-stelligen Verknüpfungen lassen sich alle allein mit den Verknüpfung 
\enquote{$\lnor$} beziehungsweise \enquote{$\lnand$} darstellen.
\end{Bemerkung}

Beweis: Es gelten die nachfolgenden Äquivalenzen, die leicht nachgerechnet 
werden können. 
\begin{align}
  \neg A    &\equiv (A \lnand A) 
    \equiv (A \lnor A) \\
    \nonumber \\
  A \land B &\equiv (A \lnand B) \lnand (A \lnand B) 
    \equiv (A \lnor A) \lnor (B \lnor B) \\
    \nonumber \\
  A \lor B  &\equiv (A \lnand A) \lnand (B \lnand B) 
    \equiv (A \lnor B) \lnor (A \lnor B) \\ 
    \nonumber \\
   A \rightarrow B &\equiv A \lnand (A \lnand B) 
     \equiv (B \lnor (A \lnor B)) \lnor (B \lnor (A \lnor B))
\end{align}

Diese Äquivalenzen können beispielsweise mittels Wahrheitstafeln bewiesen 
werden. Dies wird anhand der ersten Äquivalenz ausgeführt, siehe Tabelle
\ref{tbl:Beweis not(A) = A nand A = A nor A}.
 
\begin{table}[htbp]
\begin{center}
\begin{tabular}{c||c|c|c}
  $A$ & $\neg A$ & $A \lnand A$ & $A \lnor A$ \\ \hline
  $w$ & $f$      & $f$          & $f$       \\
  $f$ & $w$      & $w$          & $w$       \\
\end{tabular}
\caption{Beweis $\neg A \equiv A \lnand A \equiv A \lnor A$}
\label{tbl:Beweis not(A) = A nand A = A nor A}
\end{center} 
\end{table}

Es kann aber auch ohne Wahrheitstafeln bewiesen werden, durch äquivalente
Umformungen.

\begin{align*}
  (A \lnand B) \lnand (A \lnand B) \equiv \neg (A \lnand B)
  \equiv \neg (\neg (A \land B)) \equiv A \land B
\end{align*}

Die Ausführung des Beweises der restlichen Äquivalenzen kann als einfache 
Übung durchgeführt werden.
\end{Unit}

%%------------------------------------------------------------------------------
%% Abschnitt: Aussagenlogik
%%------------------------------------------------------------------------------
%%
\section{Aussagenlogik}
\label{sec:ElemLogik-Aussagenlogik}

%% -----------------------------------------------------------------------------
\begin{Unit}[Definition Tautologie, Kontradiktion]
Der Wahrheitswert einer aussagenlogischen Formel lässt sich eindeutig aus den
Wahrheitswerten der Teilaussagen ermitteln. Wenn $A$ eine beliebige Aussage 
ist, dann ist die Verknüpfung $A \lor \neg A$ stets wahr, während die 
Verknüpfung $A \land \neg A$ stets falsch ist, unabhängig vom Wahrheitswert 
von $A$. Dies führt zur nachfolgenden Definition.

\begin{Definition}
Eine aussagenlogische Formel heißt eine \Begriff{Tautologie}, wenn sie stets 
eine wahre Aussage ist. Eine aussagenlogische Formel heißt eine
\Begriff{Kontradiktion}, wenn sie stets eine falsche Aussage ist.

Ein aussagenlogische Formel heißt \Begriff{erfüllbar}, wenn es mindestens 
eine Belegung mit Wahrheitswerten gibt, so dass die Aussage wahr ist. 
\end{Definition}
\Translation{Tautologie}{tautology}
\Translation{Kontradiktion}{contradiction}
\Translation{erfüllbar}{satisfiable}

Allgemein gültige Aussage (Tautologien) heißen auch Sätze der Aussagenlogik. 
Sie bilden die Basis für die mathematische Logik und das Beweisen. Im 
Nachfolgenden werden einige Tautologien aufgestellt und beweisen.

Die Frage, ob eine Formel erfüllbar ist, ist ein tiefgehendes Problem der
(theoretischen) Informatik, das Erfüllbarkeitsproblem der Aussagenlogik.
\end{Unit}

%% -----------------------------------------------------------------------------
\begin{Unit}[Satz vom ausgeschlossenen Dritten]
Es kann nur eine Aussage $A$ oder die Negation von $A$ gelten, nichts 
anderes. 

\begin{Satz}
Es sei $A$ eine Aussage. Dann ist 
\begin{align}
  A \lor \neg A
\end{align}
eine Tautologie, der \Begriff{Satz vom ausgeschlossenen Dritten}.
\end{Satz}
\Translation{Satz vom ausgeschlossenen Dritten}{law of excluded middle}

Der Beweis kann mit Hilfe einer Wahrheitstafel (siehe Tabelle 
\ref{tbl:Beweis Satz vom ausgeschlossenen Dritten}) erbracht werden.

\begin{table}[htbp]
\begin{center}
\begin{tabular}{c||c|c|c}
  $A$ & $A$  & $\lor$ & $\neg A$ \\  \hline
  $w$ & $w$  & $w$  & $f$      \\
  $f$ & $f$  & $w$  & $w$      \\  \hline
    & 0. & 2. & 1.     \\
\end{tabular}
\caption{Beweis Satz vom ausgeschlossenen Dritten}
\label{tbl:Beweis Satz vom ausgeschlossenen Dritten}
\end{center}
\end{table} 
\qed

Somit gibt es bei der zweiwertigen Logik, die hier betrachtet wird, nur die 
Werte $wahr$ und $falsch$.
\end{Unit}

%% -----------------------------------------------------------------------------
\begin{Unit}[Satz vom Widerspruch]
\label{satz:Aus-SatzVomWiderspruch}
Es kann nicht gleichzeitig eine Aussage $A$ und die Negation davon wahr sein. 

\begin{Satz}[\Begriff{Satz vom Widerspruch}] 
Es sei $A$ eine Aussage. Dann ist
\begin{align}
  \neg (A \land \neg A)
\end{align}
eine Tautologie, der \Begriff{Satz vom Widerspruch}. 
\end{Satz}
\Translation{Satz vom Widerspruch}{law of noncontradiction}

Die Aussage $( A \land \neg A)$ ist eine Kontradiktion. Dies bedeutet, dass 
eine Aussage nicht gleichzeitig $wahr$ und $falsch$ sein kann.
\end{Unit}

%% -----------------------------------------------------------------------------
\begin{Unit}[Satz - Assoziativ-, Kommutativ- und Distributivgesetz]
Aus dem Rechnen mit Zahlen sind die Assoziativ-, Kommutativ- und 
Distributivgesetze bekannt. Entsprechende Regeln gibt es auch für Aussagen.

\begin{Satz}
Es seien $A$, $B$ und $C$ Aussagen. Es gelten das 
\Begriff{Assoziativgesetz}
\begin{align}
  (A \land  B) \land C \leftrightarrow A \land (B \land C) \\
  (A \lor B) \lor C \leftrightarrow A \lor (B \lor C)
\end{align}
, das \Begriff{Kommutativgesetz}
\begin{align}
  A \land B \leftrightarrow B \land A \\
  A \lor B \leftrightarrow B \lor A
\end{align}
und das \Begriff{Distributivgesetz}
\begin{align}
  A \land (B \lor C) \leftrightarrow (A \land B) \lor (A \land C) \\
  A \lor (B \land C) \leftrightarrow (A \lor B) \land (A \lor C) .
\end{align}
\end{Satz}
\Translation{Assoziativgesetz}{associative property}
\Translation{Kommutativgesetz}{commutative property}
\Translation{Distributivgesetz}{distributive property}

Das Assoziativgesetz ermöglicht es, einfach $A \land B \land C$ 
beziehungsweise $A \lor B \lor C$ zu schreiben, ohne dass Klammern 
gesetzt werden müssen. Dies erleichtert oftmals die Schreibarbeit. Das 
Assoziativgesetz gilt auch für mehr als drei Aussagen. Diese drei Gesetze 
erinnern stark an die entsprechenden Gesetze bei der Arithmetik mit 
\enquote{$+$}\ und \enquote{$\cdot$}, doch beim Distributivgesetz ist 
zu sehen, dass es Unterschiede gibt.
\end{Unit}

%% -----------------------------------------------------------------------------
\begin{Unit}[Satz von der doppelte Negation]
Die Negation der Negation ist wieder die ursprüngliche Aussage. 

\begin{Satz}
Es sei $A$ eine Aussage, dann ist
\begin{align} 
  \neg (\neg (A)) \leftrightarrow A
\end{align}
eine Tautologie, der \Begriff{Satz von der doppelten Negationg}. 
\end{Satz}
\Translation{Satz von der doppelten Negation}{double negation elimination}

Die doppelte Negation einer Aussage hebt sich gegenseitig auf.
\end{Unit}

%% -----------------------------------------------------------------------------
\begin{Unit}[Regeln von de Morgan]
Wie verhalten sich die Negation mit der Konjunktion und Disjunktion. 

\begin{Satz}
Es seien $A$ und $B$ Aussagen, dann gelten die \Begriff{Regeln von de Morgan}
\begin{align}
  (a)\ \neg (A \land B) \leftrightarrow \neg A \lor \neg B \\
  (b)\ \neg (A \lor B) \leftrightarrow \neg A \land \neg B
\end{align}
\end{Satz}

Die Negation einer $and$-Verknüpfung ist die $or$-Verknüpfung der Negationen 
der Aussagen. Ebenso ist die Negation einer $or$-Verknüpfung die 
$and$-Verknüpfung der Negationen. Dies sind die Regeln von de 
Morgan\index[names]{Morgan, Augustus de}.
\footnote{Augustus de Morgan (1806 - 1871), englischer Mathematiker
(\url{https://de.wikipedia.org/wiki/Augustus_De_Morgan})}

Beweis: Es wird $\neg(A \land B) \leftrightarrow (\neg(A) \lor \neg(B))$
mit Hilfe einer Wahrheitstafel (siehe 
\ref{tbl:Beweis Regel von de Morgan}) bewiesen.

\begin{table}[htbp]
\begin{center}
\begin{tabular}{c|c||c|c|c|c|c|c}
  $A$ &   $B$ & $\neg$ & $(A \land B)$ & $\leftrightarrow $ & $\neg A$ & 
    $\lor$ & $\neg B$ \\ 
    \hline
  $w$ &   $w$ & $f$   &   $w$  & $w$  &   $f$  & $f$  &   $f$  \\
  $w$ &   $f$ & $w$   &   $f$  & $w$  &   $f$  & $w$  &   $w$  \\
  $f$ &   $w$ & $w$   &   $f$  & $w$  &   $w$  & $w$  &   $f$  \\
  $f$ &   $f$ & $w$   &   $f$  & $w$  &   $w$  & $w$  &   $w$  \\  \hline
    &   &   2.  & 1. & 3. & 1. & 2. & 1. \\
\end{tabular}
\caption{Beweis Regel von de Morgan}
\label{tbl:Beweis Regel von de Morgan}
\end{center}
\end{table}

Durch ersetzen von $A$ durch $\neg A$ und $B$ durch $\neg B$ in der obigen
Äquivalenz, ergibt sich, dabei hilft auch der Satz von der doppelten 
Verneinung,
\begin{align}
  \neg(\neg A \land \neg B) \equiv (\neg(\neg A) \lor \neg(\neg B)) 
  \equiv A \lor B \ .
\end{align}
Negation auf beiden Seiten und nochmalige Anwendung des Satzes der 
doppelten Verneinung ergibt die zweite Regel von de Morgan. 

\begin{tabular}{cll}
  & $\neg(A \lor B)$ & \\
  $\equiv$ & & (doppelte Verneinung) \\
  & $\neg(\neg(\neg A) \lor \neg(\neg B))$ & \\
  $\equiv$ & & (Regel von de Morgan, a) \\
  & $\neg(\neg(\neg A \land \neg B))$ & \\
  $\equiv$ & & (doppelte Verneinung) \\
  & $\neg A \land \neg B$ & \\
\end{tabular}

\qed
\end{Unit}

%% -----------------------------------------------------------------------------
\begin{Unit}[Sätze für Beweisverfahren]
Im nachfolgenden werde einige Sätze der Aussagenlogik vorgestellt, die bei 
den Beweisverfahren immer wieder die Grundlage bilden.
\end{Unit}

%% -----------------------------------------------------------------------------
\begin{Unit}[Kontraposition]
Wenn aus einer Aussage $A$ die Aussage $B$ folgt, dann ist das äquivalent zu
der Folgerung von nicht $B$ (also Aussage $B$ gilt nicht) zu nicht $A$.

\begin{Satz} Es seien $A$ und $B$ beliebige Aussagen, dann gilt die
\Begriff{Kontrapositions} 
\begin{align}
  (A \rightarrow B) \leftrightarrow (\neg B \rightarrow \neg A) \ .
\end{align}
\end{Satz}
\Translation{Kontrapositions}{contraposition}

Wenn die Folgerung \enquote{aus $A$ folgt $B$} bewiesen werden soll, dann 
kann dafür auch die Folgerung \enquote{aus nicht $B$ folgt nicht $A$} 
bewiesen werden.
\end{Unit}

%% -----------------------------------------------------------------------------
\begin{Unit}[Satz - Abtrennungsregel]
Es wird der Zusammenhang zwischen der Folgerung \enquote{$A \rightarrow B$} 
und der Aussage $A$ betrachtet.

\begin{Satz} Es seien $A$ und $B$ beliebige Aussagen, dann gilt der 
\Begriff{Satz zum modus ponens} oder die \Begriff{Abtrennungsregel}
\begin{align}
  ((A \rightarrow B) \land A) \rightarrow B \ .
\end{align}
\end{Satz}
\Translation{Abtrennungsregel}{modus ponens}

Wenn die Folgerung \enquote{aus $A$ folgt $B$} wahr ist und die Aussage $A$ 
wahr ist, dann ist auch die Aussage $B$ wahr.
\end{Unit}

%% -----------------------------------------------------------------------------
\begin{Unit}[Satz zum modus tollens]
Es wird der Zusammenhang zwischen der Folgerung \enquote{$A \rightarrow B$} 
und der Aussage $B$ betrachtet.

\begin{Satz} Es seien $A$ und $B$ beliebige Aussagen, dann gilt der 
\Begriff{Satz zum modus tollens} 
\begin{align}
  (A \rightarrow B) \land \neg B \rightarrow \neg A \ .
\end{align}
\end{Satz}
\Translation{Satz zum modus tollens}{modus tollens}

Wenn die Folgerung \enquote{aus $A$ folgt $B$} wahr ist und die Aussage $B$ 
nicht wahr ist, dann ist auch die Aussage $A$ nicht wahr. Wäre die Aussage 
$A$ wahr, dann wäre nach der Abtrennungsregel die Aussage $B$ wahr.

Beweis: 
Der Beweis ist als Wahrheitstabelle in der Tabelle \ref{tbl:Beweis Satz zum
modus tollens} zu sehen.

\begin{table}[htbp]
\begin{center}
\begin{tabular}{c|c||c|c|c|c|c|c}
  $A$ &   $B$ & $(A \rightarrow B)$ & $\land$ & $\neg B$ & $\rightarrow$ 
    & $\neg A$ \\ \hline
  $w$ &   $w$ & $w$   &   $f$  & $f$  & $w$  &  $f$  \\
  $w$ &   $f$ & $f$   &   $f$  & $w$  & $w$  &  $f$  \\
  $f$ &   $w$ & $w$   &   $f$  & $f$  & $w$  &  $w$  \\
  $f$ &   $f$ & $w$   &   $w$  & $w$  & $w$  &  $w$  \\  \hline
    &   &   1.  & 2. & 1. & 3. & 1. \\
\end{tabular}
\caption{Beweis Satz zum modus tollens}
\label{tbl:Beweis Satz zum modus tollens}
\end{center}
\end{table} 

Der Beweis kann jedoch auch ohne Wahrheitstafel, durch äquivalente 
Umformungen erfolgen.

\begin{tabular}{c l l}
    & $(A \rightarrow B) \land \neg B$ & \\
  $\equiv$ &  & (siehe Bemerkung \ref{bem:Aus-FolgerungVereinfachen}) \\
   & $(\neg A \lor\ B) \land \neg B$ \\
  $\equiv$ &  & Distributivgesetz \\
   & $(\neg A \land \neg B) \lor (B \land \neg B)$ &  \\
  $\equiv$ &  & (Satz \ref{satz:Aus-SatzVomWiderspruch} vom Widerspruch) \\
   & $(\neg A \land \neg B) \lor false$ &  \\
  $\equiv$ &  &  (siehe Bemerkung \ref{bem:Aus-FolgerungVereinfachen}) \\
   & $\neg A \land \neg B$ &  \\
  $\Rightarrow$ &  &  \\
   & $\neg A$ &  \\
\end{tabular}

Beim letzten Schritt keine Äquivalenz sondern nur eine Folgerung!
\qed 
\end{Unit}

%% -----------------------------------------------------------------------------
\begin{Unit}[Satz - Kettenschlussregel] 
Was ist, wenn es mehrere Folgerungen gibt?

\begin{Satz} Es seien $A$, $B$ und $C$ beliebige Aussagen, dann gilt der 
\Begriff{Satz zum modus barbara} oder die \Begriff{Kettenschlussregel}
\begin{align}
  [(A \rightarrow B) \land (B \rightarrow C)] \rightarrow (A \rightarrow C) 
  \ .
\end{align}
\end{Satz}
\Translation{Kettenschluss}{chain inference}
\Translation{modus barbara}{modus barbara}

Wenn die Folgerung \enquote{aus $A$ folgt $B$} wahr ist und auch die 
Folgerung \enquote{aus $B$ folgt $C$} wahr ist, dann ist auch die 
Folgerung \enquote{aus $A$ folgt $C$} eine wahre Aussage.

\end{Unit}

%%------------------------------------------------------------------------------
%% Abschnitt: Prädikatenlogik
%%------------------------------------------------------------------------------
%%
\section{Prädikate}
\label{sec:ElemLogik-Praedikate}

%% =============================================================================
%\subsection*{Aussageformen}

%% -----------------------------------------------------------------------------
\begin{Unit}[Beispiel]
\label{bsp:ausPraedikate}
Es gibt Formulierungen, denen kein Wahrheitswert direkt zugeordnet werden 
kann. Die nachfolgenden Sätze sind keine Aussage.

\begin{itemize}
\item $P(x)$ := \enquote{$x$ ist eine Primzahl.} 
\item $T(x, y)$ := \enquote{$x$ ist ein Teiler von $y$.} 
\item $S(x, y, z)$ := \enquote{$x$ ist die Summe von $y$ und $z$}
\end{itemize}

Wenn jedoch für die Platzhalter $x$, $y$ oder $z$ konkrete Werte aus einem
Grundbereich von Werten einsetzt werden, dann ergeben sich hieraus Aussagen, 
denen ein Wahrheitswert zuordnen werden kann.
\end{Unit}

%% -----------------------------------------------------------------------------
\begin{Unit}[Definition Variable, Aussageform] Die Beispiele führen zur
nachfolgenden Definition.

\begin{Definition}
Eine \Begriff{Variable} über einem Grundbereich ist ein Symbol, für das 
spezielle Objekte eines Grundbereichs eingesetzt werden können. Eine mit 
Hilfe mindestens einer Variablen ausgedrückte Formulierung heißt
\Begriff{Aussageform} oder \Begriff{Prädikat}, wenn beim Einsetzen von 
bestimmten Objekten des Grundbereichs für die Variable(n) eine Aussage 
entsteht.
\end{Definition}
\Translation{Variable}{quantified variables}
\Translation{Aussageform}{propositional formula}
\Translation{Prädikat}{predicate}
\Translation{Prädikatenlogik}{first-order logic}

Der Grundbereich wird manchmal auch \Begriff{Universium} genannt.
\Translation{Universum}{universe}
\end{Unit}

%% -----------------------------------------------------------------------------
\begin{Unit}[Beispiel]
Für die Aussageformen in Beispiel \ref{bsp:ausPraedikate} gelten, wenn 
konkrete Werte eingesetzt werden:
\begin{itemize}
  \item $w(P(3)) = w$; $w(P(4)) = f$; $w(P(5)) = w$
  \item $w(T(2,3)) = f$; $w(T(2,4)) = w$
  \item $w(S(5,2,3)) = w$
\end{itemize}

Durch die Einsetzung von konkreten Werten aus dem Grundbereich in die 
Aussageformen werden konkrete Aussagen gebildet, für die jeweils ein 
Wahrheitswert bestimmbar ist.
\end{Unit}

%% -----------------------------------------------------------------------------
\begin{Unit}[Beispiel]
In Programmiersprachen gibt es den primitiven Datentyp Wahrheitswert
(\textbf{\texttt{bool}}, \textbf{\texttt{boolean}} mit den beiden möglichen 
Werte \enquote{\textbf{\texttt{true}}} und \enquote{\textbf{\texttt{false}}} 
oder \enquote{\textbf{\texttt{True}}} und \enquote{\textbf{\texttt{False}}}. 
Dies ist eine Darstellung der Wahrheitswerte. Eine Aussage ist dann einfach 
eine boolesche Variable. Bei der Programmierung gibt es viele Aussageformen. 
Jede Methode mit einem Wahrheitswert als Rückgabewert ist Aussageform:
\end{Unit}

%% -----------------------------------------------------------------------------
\begin{Unit}[Anmerkung]
Für die Formulierung mathematischer Theorien reicht die Aussagenlogik noch 
nicht aus. In der Aussagenlogik werden Aussagen betrachtet. In der 
Prädikatenlogik werden Prädikate oder Aussageformen betrachtet.

Bei der weiteren Analyse von Aussagen stößt werden Subjekte, Prädikate und
sogenannte quantifizierbare Redeteile wie \enquote{für alle}\ und 
\enquote{es gibt} betrachtet. Subjekte sind Namen für Dinge aus einer 
bestimmten vorgegebenen Individualmenge, Prädikate sind Namen für 
Relationen\footnote{Relationen werden erst später genauer betrachtet} auf 
dieser Individualmenge. 1-stellige Prädikate heißen Eigenschaften. Weiter 
werden die Quantoren \enquote{$\forall$} (\enquote{für alle} oder 
Generalisator) und \enquote{$\exists$} (\enquote{es gibt} oder 
Partikularisator) eingeführt. Darüber hinaus gibt es noch das
Gleichheitszeichen \enquote{=} und Vergleichsoptionen \enquote{<} und 
\enquote{>}, um Identitäten darstellen zu können. Weiter werden noch 
Funktionen\footnote{Funktionen werden erst später genauer betrachtet} 
(in $n$ Variablen) eingeführt, die jedoch genau genommen nur 
(($n+1$)-stellige)  Relationen sind. Damit sind Elemente der
\Begriff{Prädikatenlogik} der  1. Stufe mit Identität eingeführt. 
Damit können mathematische Aussagen formulieren werden. Dies und die 
Erweiterung in höhere Stufen wird hier nicht weiter ausgeführt. 
\end{Unit}

%% -----------------------------------------------------------------------------
\begin{Unit}[Definition All-Quantor, Existenz-Quantor]
Generalisator und Partikularisator werden wie folgt definiert.

\begin{Definition}
Es sei $A(x)$ eine Aussageform mit einer Variablen $x$ in einem Universum $U$. 

\begin{itemize}
\item
Die \Begriff{All-Aussage}
\begin{align}
  \left( \forall x \in U \right) : \left[ A(x) \right]
\end{align}
ist genau dann wahr, wenn es für jedes $x$ aus $U$ die Aussage $A(x)$ 
wahr ist. Das Symbol $\forall$ heißt \Begriff{All-Quantor} und wird 
\enquote{für alle} gelesen.
\item
Die \Begriff{Existenz-Aussage}
\begin{align}
  \left( \exists x \in U \right) : \left[ A(x) \right]
\end{align}
ist genau dann wahr, wenn es (mindestens) ein $x \in U$ gibt, so dass die 
Aussage $A(x)$ wahr ist. Das Symbol $\exists$ heißt 
\Begriff{Existenz-Quantor} und wird \enquote{es gibt} gelesen.
\end{itemize}
\end{Definition}
\Translation{All-Aussage}{universal quantification}
\Translation{Existenz-Aussage}{existential quantification}

Beim Existenz-Quantor muss es mindestens ein $x \in U$ geben. Es können auch 
viele existieren. Durch Ergänzungen der Form $\exists^n$, $\exists^{\geq n}$, 
$\exists^{\leq n}$ oder $\exists^{n \ldots m}$ kann ausgedrückt werden, dass 
genau $n$, mindestens $n$, höchstens $n$ oder zwischen $n$ und $m$ (inklusive 
der Grenzen) Elemente mit einer Eigenschaft gibt.

Manchmal wird kurz $\forall x \in U: A(x)$ oder $\exists x \in U: A(x)$ 
geschrieben, lässt also manche Klammern weg, ohne dass es zu
Missverständnissen kommt. Wenn das Universum klar ist, dann wird das Universum
nicht expliziz aufgeführt.

All-Quantoren und Existenz-Quantoren können auch mehrfach hintereinander 
auftreten.
\end{Unit}

%% -----------------------------------------------------------------------------
\begin{Unit}[Beispiel]
Gegeben sei die Aussage \enquote{Es existiert (mindestens) eine Primzahl 
größer 5 und kleiner 9}. \\
Sei $P$ das 1-stellige Prädikat (Eigenschaft) \enquote{ist Primzahl}, dann 
lässt sich die Aussage kurz in mathematischer Schreibweise darstellen als: 
\begin{align*}
  \exists n \in \NN :  P(n) \land (5 < n < 9)  \quad .
\end{align*}
\end{Unit}

%% -----------------------------------------------------------------------------
\begin{Unit}[Beispiel]
Die Aussage \enquote{Für alle reelle Zahlen $x$ größer 0 existiert 
(mindestens) eine natürliche Zahl $n$, so dass $1/n$ kleiner $x$ ist} lässt 
sich in mathematischer Form darstellen als:
\begin{align*}
  \forall (x \in \RR^+) : \exists n \in \NN : \frac{1}{n} < x \quad .
\end{align*}
\end{Unit}

%% -----------------------------------------------------------------------------
\begin{Unit}[Satz Verneinung]
Einfache Aussagen können negiert werden. Auch die All-Aussage und die 
Existenz-Aussage können verneint werden. 

\begin{Satz} Die All-Aussage und die Existenz-Aussage können verneint werden
\begin{itemize}
\item Gegeben sei die All-Aussage $\forall x \in U : A(x)$, \enquote{für 
alle $x \in U$ gilt $A(x)$}. Dann gilt
  \begin{align}
    \neg \left( \forall x \in U \right) : A(x) \equiv \exists x \in U : 
    \neg A(x) \quad .
  \end{align}
  Somit ist die Negation der obigen All-Aussage: \enquote{es existiert ein 
  $x \in U$, so dass $A(x)$ nicht gilt}.
\item Gegeben sei die Existenz-Aussage $\exists x \in U : A(x)$, 
\enquote{es existiert ein $x \in U$, so dass $A(x)$ gilt}. Dann gilt
  \begin{align}
    \neg (\exists x \in U) : A(x) \equiv \forall x \in U : \neg A(x) \quad .
  \end{align}
  Somit ist die Negation der obigen All-Aussage: \enquote{für alle $x \in U$ 
  ist die Negation von $A(x)$ wahr}.
\end{itemize}
\end{Satz}

Bei mehrfach hintereinander kommen All-Quantoren und Existenz-Quantoren kann 
die Negation manchmal etwas komplexer sein.
\end{Unit}

%%------------------------------------------------------------------------------