%% =============================================================================
%% Mathematik
%% Kapitel 07 - Homomorphismen
%% Andreas Zeh-Marschke
%% =============================================================================

\chapter{Strukturen und strukturerhaltende Abbildungen}
\label{cha:Gdl-K07-Homomorphismen}

Dieses Kapitel ist wieder eher informeller Art. Strukturen und 
strukturerhaltende Abbildungen sind wichtig, allgemeine Begriffe in der 
Mathematik. Es gibt vielfältige Arten von Strukturen: Gruppen, Ringe, Körper,
Vektorräume, Verbände, Graphen um nur einige wenige Beispiele zu nennen.

%% -----------------------------------------------------------------------------
\begin{unitAnmerkung}[Strukturen]
Auf Mengen werden oftmals Operatoren definiert, Das sind Abbildungen von 
Elementen der Menge auf andere Elemente der Menge. Das kann die Addition von 
zwei natürlichen Zahlen sein, deren Ergebnis wieder eine natürliche Zahl ist.
Das kann die skalare Multiplikation eines Vektors sein. Das Ergebnis ist wieder
ein Vektor. Ebenso kann es Relationen geben. Beispielsweise die 
Kleiner-Relation $<$ bei den natürlichen Zahlen. Genauso gibt es spezielle
Konstanten in den Mengen. Beispielsweise die $0$ bei den reellen Zahlen als
das neutrale Element bei der Addition oder die $1$ als neutrales Element der
Multiplikation.

Eine Menge mit Operationen, Relationen und Konstanten nennt man eine 
\Begriff{Struktur}. Es gibt viele Strukturen, die klar definiert werden. Es 
kann auch verschiedene Mengen geben, welche die selbe Struktur haben. Die
Untersuchung der Strukturen und die dabei gewonnenen Erkenntnisse gelten dann 
für alle Mengen mit derselben Struktur. 
\end{unitAnmerkung}

%% -----------------------------------------------------------------------------
\begin{unitAnmerkung}[Unterstrukturen]
Es sei $M$ eine Menge mit einer Strukturen. Das bedeutet, eine Menge mit 
Operationenn $f_i$, mit Relationen $R_j$ und Konstanten $C_k$. Ist die 
Teilmenge $U \subseteq M$ bezüglich der Operationen abgeschlossen und besitzt 
sie dieselben Konstanten wie $M$, dann ist $U$ eine \Begriff{Unterstruktur} 
von $M$. Für die Relationen gibt es keine weitere Anforderung. 
\end{unitAnmerkung}

%% -----------------------------------------------------------------------------
\begin{unitAnmerkung}[Homomorphismen]
Es seien für $n = 1,2$ $M_n$ Strukturen  mit Operationen $f_n,i$, Relationen 
$R_n,j$ und Konstanten $C_n,k$. Die Anzahl der Operationen, die Anzahl der
Relationen und die Anzahl der Konstanten sei dabei identisch, genauso die 
Stelligkeit der Operationen und Relationen. Eine Abbildung $\phi$ von $M_1$ 
nach $M_2$ heißt ein \Begriff{Homomorphismus}, wenn für alle Operationen, 
Relationen und Konstanten gilt:
\begin{align}
  &\phi(f_{1,i}(a_1, \ldots, a_{m_i}) 
  = f_{2,i}(\phi(a_1), \ldots , \phi(a_{m_i}) \ ,\\
  &\phi(R_{1,j}(b_1, \ldots, b_{l_j}) 
  = R_2,j(\phi(b_1), \ldots , \phi(b_{l_j}) \ \text{und}\\
  &\phi(C_{1,p}) = C_{2,p}\ .
\end{align}
Die Operationen, Relationen und Konstanten werden durch die Abbildung $\Phi$
übertragen, so dass diese weiterhin gültig sind.

Ist die Abbildung injektiv, dann heißt die Abbildung ein 
\Begriff{Monomophismus}.
Ist die Abbildung surjektiv, dann heißt die Abbildung ein 
\Begriff{Epimophismus}.
Ist die Abbildung bijektiv, dann heißt die Abbildung ein 
\Begriff{Isomophismus}.

Ist $M_1 = M_2$, dann heißt die Abbildung ein \Begriff{Endomophismus}.
Ein bijektiver Endomorphismus heißt \Begriff{Automorphismus}
\end{unitAnmerkung}
