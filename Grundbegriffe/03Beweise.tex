%% =============================================================================
%% Mathematische Grundlagen - Grundbegriffe
%% Kapitel 03 - Beweise
%% Autor: Andreas Zeh-Marschke
%% Datum: 2025-03-07
%% =============================================================================

\chapter{Beweisverfahren}
\label{cha:Gdl-K03-Beweise}

%\begin{unit}
In diesem Kapitel werden verschiedene Beweisverfahren vorgestellt und an 
ersten kleinen Beispielen demonstriert. Im weiteren Verlauf wird das eine oder
andere Verfahren eingesetzt. 

Unter einem \Begriff{Beweis} wird die Ableitung einer Aussage aus anderen 
Aussagen nach bestimmten, logischen Schlussregeln verstanden. Es werden hier 
einige der Beweisverfahren an konkreten Beispielen betrachtet. Ziel eines 
Beweises ist es immer, aus einer Menge von wahren Aussagen durch logische
Schlussfolgerungen eine Behauptung zu bestätigen. 

%\end{unit}
\Translation{Beweis}{proof}

%%------------------------------------------------------------------------------
\section{Gegenbeispiel}
\label{sec:Beweis - Beweisverfahren}

%% -----------------------------------------------------------------------------
\begin{Unit}[Gegenbeispiel]
Nicht jede Behauptung ist wahr. Die Anführung eines einzigen 
\Begriff{Gegenbeispiel}es ist eine Möglichkeit, eine Behauptung zu widerlegen. 
Eine Möglichkeit, die immer erwägt werden sollte. Für die 
Widerlegung einer Aussage genügt ein einziges Gegenbeispiel. Für einen Beweis 
einer Aussage reichen auch viele Beispiele nicht. 
\end{Unit}
\Translation{Gegenbeispiel}{counterexample}

%%------------------------------------------------------------------------------
\begin{Unit}[Beispiele] \ 

\begin{enumerate}
\item Die Behauptung, dass alle Primzahlen ungerade Zahlen sind ist schnell 
  widerlegt. Das Gegenbeispiel der Primzahl 2 reicht aus. 
\item Auch die Behauptung, dass alle ungeraden Zahlen Primzahlen sind, ist 
  durch das Beispiel 1 oder durch das Beispiel 9 oder durch das Beispiel 121 
  schnell widerlegt.
\item Die Goldbachsche Vermutung\footnote{ 
  \url{https://de.wikipedia.org/wiki/Goldbachsche_Vermutung}, benannt nach
  Christian Goldbach\index[names]{Goldbach, Christian} (1690-1764),
  deutscher Mathematiker,
  (\url{https://de.wikipedia.org/wiki/Christian_Goldbach)}} 
  dass jede gerade Zahl größer als 2 als Summe von zwei Primzahlen 
  geschrieben werden kann, ist bis $4 \cdot 10^{18}$ als richtig 
  nachgerechnet worden. Das sind sehr viele Beispiele, aber kein Beweis.
\end{enumerate}
\end{Unit}
\Translation{Goldbachsche Vermutung}{Goldbach's conjecture}

%%------------------------------------------------------------------------------
\section{Vollständiges Durchrechnen}
\label{sec:Beweis - Durchrechnen}

%% -----------------------------------------------------------------------------
\begin{Unit}[Vollständiges Durchrechnen]
Ebenfalls eine einfache Möglichkeit ist das \Begriff{vollständige Durchrechnen} 
\index{Durchrechnen, vollständig} aller Fälle. Dies ist jedoch nur dann 
möglich, wenn es endlich viele Fälle gibt. Es ist nur dann wirklich einfach, 
wenn die Anzahl der möglichen Fälle klein ist. Im Abschnitt 
\ref{cha:Gdl-K02-ElemLogik} wurden einige Beweise dadurch geführt. Mit dem 
Aufkommen der Computer sind derartige Beweise verstärkt aufgekommen. Ein 
berühmtes Beispiel ist der Beweis des Vier-Farben-Problem\footnote{
  Vier-Farben-Theorem (\url{https://de.wikipedia.org/wiki/Vier-Farben-Satz}): 
  Wie viele Farben reichen aus, um eine beliebige Karte so einzufärben, dass 
  je zwei aneinander grenzende Länder unterschiedliche Farben haben? Diese 
  Problem wurde von Francis Guthrie\index[names]{Guthrie, Francis} (1831-1899),
  englischer Mathematiker 
  (\url{https://de.wikipedia.org/wiki/Francis_Guthrie}), 1852 formuliert. Erst 
  1977 gelangen Kenneth Appel\index[names]{Appel, Kenneth} (1932-2013),
  US-amerikanischer Mathematiker
  (\url{https://de.wikipedia.org/wiki/Kenneth_Appel}) und Wolfgang Haken
  \index[names]{Haken, Kenneth} (1928-2022), deutsch-US-amerikanischer 
  Mathematiker (\url{https://de.wikipedia.org/wiki/Wolfgang_Haken}) mit Hilfe 
  eines Computerprogrammes der Beweis.}, 
das mit Hilfe der Reduzierung auf endliche viele Untersuchungsfälle und dem
anschließenden Durchrechnen mittels eines Computerprogramms gelöst wurde.
\end{Unit}
\Translation{Vollständiges Durchrechnen}{brute force method}
\Translation{Vier-Farben-Satz}{four-colour-theorem}

%%------------------------------------------------------------------------------
\begin{Unit}[Beispiel]

Das Aufstellen der Wahrheitstafel bei der elementaren Logik ist ein Beispiel
für das vollständige durchrechnen. Es wurden jeweils alle möglichen Belegungen 
der Aussagen mit den Wahrheitswerten berechnet.
\end{Unit}

%%------------------------------------------------------------------------------
\section{Direkter Beweis}
\label{sec:Beweis - direkt}

%% -----------------------------------------------------------------------------
\begin{Unit}[Direkter Beweis]
Beim \Begriff{direkten Beweis}\index{Beweis, direkt} wird eine Aussage aus 
einer wahren Aussage durch direkte logische Schlussregeln hergeleitet. Als 
Basis dient hierbei der Satz zum modus ponens.
\begin{align}
  ((A \rightarrow B) \land A) \rightarrow B
\end{align}
\end{Unit}
\Translation{direkter Beweis}{direct proof}
%% -----------------------------------------------------------------------------
\begin{Unit}[Beispiel]
\label{bsp:Aus-DirekterBeweis}

\textbf{Behauptung}: Es sei $n$ eine ungerade natürliche Zahl, dann ist auch 
$n^2$ eine ungerade Zahl. Hierbei sei \enquote{$n$ ist ungerade} die Aussage 
$A$ und \enquote{$n^2$ ist ungerade} die Aussage $B$.

\textbf{Beweis}: Da $n$ eine ungerade natürliche Zahl ist, existiert eine Zahl 
$k\ (\in \NN_0)$, mit $n=2k+1$. Damit gilt 
\begin{align}
  n^2 = (2k+1)^2 = 4k^2+4k+1 = 2(2k^2+2k)+1 \quad .
\end{align}
Somit ist auch $n^2$ eine ungerade Zahl. \qed

Damit wurde die Aussage direkt bewiesen.
\end{Unit}

%%------------------------------------------------------------------------------
\section{Indirekter Beweis}
\label{sec:Beweis - indirekt}

%% -----------------------------------------------------------------------------
\begin{Unit}[Indirekter Beweis]
Der \Begriff{indirekten Beweis}\index{Beweis, indirekt} basiert auf dem 
Kontrapositionssatz, der folgenden Äquivalenz:
\begin{align}
  (A \rightarrow B) \leftrightarrow (\neg B \rightarrow \neg A) 
\end{align}
Es wird somit von der Negation der Aussage $B$ ausgegangen und daraus die 
Negation der Aussage $A$ bewiesen. Auf Grund der Äquivalenz wird die Aussage
damit indirekt bewiesen. \vspace*{0.5ex}
\end{Unit}
\Translation{indirekter Beweis}{proof by contraposition infers}

%% -----------------------------------------------------------------------------
\begin{Unit}[Beispiel]
\label{bsp:Aus-IndirekterBeweis}

\textbf{Behauptung}: Ist $n^2$ eine gerade Zahl, dann ist auch $n$ eine gerade 
Zahl.

\textbf{Beweis}: $A$ sei die Aussage \enquote{$n^2$ ist gerade}, $B$ sei die 
Aussage \enquote{$n$ ist eine gerade Zahl}. Die Negation zu $B$ ist die Aussage 
\enquote{$n$ ist ungerade}. Beim Beispiel \ref{bsp:Aus-DirekterBeweis} zum 
direkten Beweis wurde gezeigt, dass dann $n^2$ eine ungerade Zahl ist, dies 
ist die Negation von $A$. Aus der Negation von $B$ folgt die Negation von $A$, 
damit folgt aus $A$ die Aussage $B$. \qed

Damit wurde die Aussage indirekt bewiesen.
\end{Unit}

%%------------------------------------------------------------------------------
\section{Beweis durch Widerspruch}
\label{sec:Beweis - Widerspruch}

%% -----------------------------------------------------------------------------
\begin{Unit}[Beweis durch Widerspruch]
Dieser Beweis basiert auf dem Satz zum modus tollens.
\begin{align}
  (A \rightarrow B) \land \neg B \rightarrow \neg A
\end{align}
Wenn aus der Aussage $A$ die Aussage $B$ folgt, jedoch die Negation von $B$ 
wahr ist, dann kann die Aussage $A$ nicht wahr sein, denn ansonsten würde auch 
die Aussage $B$ wahr sein, was ein Widerspruch darstellt. Somit ist die 
Negation von $A$ wahr. Somit ist es der \Begriff{Beweis durch Widerspruch}. 
\end{Unit}
\Translation{Beweis durch Widerspruch}{proof by contradiction}

%% -----------------------------------------------------------------------------
\begin{Unit}[Beispiel]
Ein schon über 2000 Jahre alter Beweis durch Widerspruch.

\Begriff{Behauptung}: $\sqrt{2}$ ist irrational. \\
\Begriff{Beweis}: 
Angenommen $\sqrt{2}$ ist eine rationale Zahl. Dann gibt es teilerfremde, 
natürliche Zahlen $p$ und $q$, so dass $\sqrt{2} = \frac{p}{q}$ gilt. 
Damit ergibt sich $2q^2 = p^2$. Damit ist $p^2$ gerade und damit auch 
$p$ (siehe Beispiel \ref{bsp:Aus-IndirekterBeweis} zum indirekten Beweis). 
Somit gibt es eine Zahl $k$, so dass  $p=2k$ gilt. 
Eingesetzt in die Gleichung ergibt sich $2q^2 = 4k^2$. Durch 2 kürzen ergibt 
$q^2 = 2k^2$. 
Damit ist $q^2$ gerade und somit auch $q$. 
Dies widerspricht jedoch der Annahme, dass $p$ und $q$ teilerfremd sind, also 
ist die Annahme, dass $\sqrt{2}$ rational ist falsch! 
\vspace*{0.5ex} 
\qed

Damit wurde die Aussage, dass $\sqrt{2}$ irrational ist, durch Widerspruch
bewiesen.
\end{Unit}

%%------------------------------------------------------------------------------
\section{Vollständige Induktion}
\label{sec:Beweis - Induktion}

%% -----------------------------------------------------------------------------
\begin{Unit}[Satz zu vollständigen Induktion]
Beim Beweis durch \Begriff{vollständige Induktion}\index{Induktion, 
vollständige} wird eine Aussage der Form $\forall$n $\in\ \mg{N}: A(n)$
bewiesen, indem zuerst die Aussage für $n = 1$ bewiesen wird 
(Induktionsbeginn oder Induktionsverankerung). In der Induktionsannahme wird
angenommen, dass $A(n)$ für ein $n \in \NN$ gilt. Im Induktionsschritt von 
$n$ auf $n+1$ wird die Aussage $A(n+1)$, basierend auf der Aussage $A(n)$ 
bewiesen: ($A(n)$ $\rightarrow$ $A(n+1)$). Durch wiederholte Anwendung des 
Satzes des modus ponens gilt $A(n)$ für alle $n\ \in\ \mg{N}$. 

\begin{Satz}
Gilt die Aussage $A(0)$ und gilt $\forall n \in \NN: A(n) \rightarrow A(n+1)$,
dann ist $A(n)$ für alle $n \in \NN$ wahr.
\end{Satz}

Manchmal wird nicht bei $n = 0$ oder $n = 1$ begonnen, sondern bei einem 
höheren Wert. Auch wird beim Induktionsschritt manchmal nicht nur $A(n)$,
sondern auch $A(n-1), A(n-2), \ldots$ verwendet.
\end{Unit}
\Translation{vollständige Induktion}{mathematical induction}

%% -----------------------------------------------------------------------------
\begin{Unit}[Beispiel]
\ 

\textbf{Behauptung}:
\begin{align}
  \forall n\in \mg{N}:\ \sum_{i=1}^n i = \frac{n(n+1)}{2}
\end{align}
\textbf{Beweis}: Die Aussage A(n) ist für $n \in \mg{N}$
\begin{align}
  A(n)\ :\ \sum_{i=1}^n i = \frac{n(n+1)}{2}
\end{align}
Induktionsbeginn: Es gilt A(1), denn $\sum_{i=1}^1 i = 1 = 
\frac{1(1+1)}{2}$. \\[1ex]
Induktionsannahme: A(n) gilt für ein $n \in \NN$. \\[1ex]
Induktionsschritt von $n$ auf $n+1$:
\begin{align}
  \sum_{i=1}^{n+1} i = \sum_{i=1}^n i\ +(n+1) = \frac{n(n+1)}{2} + (n+1) 
  = \frac{(n+1)(n+2)}{2}
\end{align}
Beim ersten Gleichheitszeichen wurde der Term für $n+1$ so umgeformt, dass ein
Term entsteht, für den die Induktionsannahme verwendbar ist. Beim mittleren 
Gleichheitszeichen wurde die Induktionsannahme verwendet.
\end{Unit}

%% -----------------------------------------------------------------------------
\begin{Unit}[Induktive Definition]
Es gibt nicht nur das induktive Beweisen, sondern auch das induktive Definieren.
Hierbei werden Glieder eine Folge durch vorherige Werte definiert. Somit 
entsteht eine unendliche Folge. Ausgangspunkt ist ein Wert für $a_0$. Der Wert
$a_1$ wird dann mit Hilfe von $a_0$ definiert. Die Werte $a_{i+1}$ werden dann
durch $a_i$ definiert.

Es kann auch mehr als ein Startwert definiert werden und die Definition des 
Folgegliedes kann sich auf mehrere vorherige Werte beziehen.
\end{Unit}

%% -----------------------------------------------------------------------------
\begin{Unit}[Beispiel Fakultät]
Die \Begriff{Fakultät} $n!$ kann induktiv definiert werden: $0! = 1$ und 
$(n+1)! = n! \cdot (n+1)$. Damit ergeben sich $1! = 1$, $2! = 2$, $3! = 6$, 
$4! = 24$, $\ldots$. 
\end{Unit}

%% -----------------------------------------------------------------------------
\begin{Unit}[Beispiel Fibonacchi-Zahlen]
Die \Begriff{Fibonacchi-Zahlen}\index[names]{Fibonacci}
\footnote{Fibonacci, auch Leonardo von Pisa, (um 1170-nach 1240), 
italienischer Mathematiker, 
\url{https://de.wikipedia.org/wiki/Leonardo_Fibonacci}} 
werden mit Hilfe von zwei Basiszahlen definiert: $f_0 = 0$ und $f_1 = 1$. 
Die Vorschrift für die Definition lautet für $i \in \NN_0$: $f_{i+2} = f_{i+1} 
+ f_i$. Damit ergibt sich für die Folge der Fibonacci-Zahlen:
\begin{align*}
  0, 1, 1, 2, 3, 5, 8, 13, 21, 34, 45, \ldots \ .
\end{align*}
\end{Unit}
%%------------------------------------------------------------------------------
