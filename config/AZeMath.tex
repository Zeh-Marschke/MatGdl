%% =============================================================================
%% AZeMath
%% Grundlegende Festlegungen für LaTeX-Datei für Artikel, Report und
%% Buch für Mathematische Sachverhalte
%% -----------------------------------------------------------------------------
%% AZeRahmen, Version 3.0.01 - 2025-01-25
%% =============================================================================

%% -----------------------------------------------------------------------------
\usepackage{amssymb}          %% mathematische Symbole
\usepackage{amsmath}          %% Mathematikumgebung von AMS
\usepackage{amsfonts}         %% mathematische Schriftfamilien
\usepackage{framed}
\usepackage{bbm}              % mathematische Mengensymbole

\usepackage[amsmath,thmmarks,amsthm,framed]{ntheorem}

% --- 2025-04-20 --- 
%\usepackage{chngcntr}
\newtheorem{Aufgabe}{Aufgabe}[chapter]
\newtheorem{Loesung}{Lösung}[section]
%\counterwithout{Loesung}{chapter}

% --- Zerlegen in einzelnen Einheiten, wobei die Nummerierung vorne ist
\theoremstyle{change}
\theorembodyfont{\normalfont}
\newtheorem{Unit}{}[section]
% --- 2025-04-20 --- 
%\usepackage{chngcntr}
%\newtheorem{UnitAfg}{Aufgabe}[chapter]
%\newtheorem{UnitLsg}{Lösung}[section]
%\counterwithout{UnitLsg}{chapter}

%--- Definition von bunt gerahmten Theormen (ohne Nummern)
\theoremstyle{nonumberplain}
\def\theoremframecommand{\fcolorbox{green}{white}}
\newshadedtheorem{Definition}{Definition}
\def\theoremframecommand{\fcolorbox{red}{white}}
\newshadedtheorem{Theorem}{Theorem}
\newshadedtheorem{Satz}{Satz}
\def\theoremframecommand{\fcolorbox{blue}{white}}
\newshadedtheorem{Lemma}{Lemma}
\newshadedtheorem{Proposition}{Proposition}
\newshadedtheorem{Bemerkung}{Bemerkung}
\newshadedtheorem{Korollar}{Korollar}
\newshadedtheorem{Folgerung}{Folgerung}


\usepackage{gauss}            % Symbole für Gauss-Algorithmus
  \def\rowaddtolabel#1{\scriptstyle #1}  %% Faktoren bei Quelle und Ziel
\usepackage{polynom}          % Polynomdivision
%% -----------------------------------------------------------------------------

%% -----------------------------------------------------------------------------
%% mathematische Befehle 
%% -----------------------------------------------------------------------------
\newcommand*{\eoe}{\hfill $\Delta$}     % end of example
\newcommand*{\dd}[1]{\,\mathrm{d}#1}    % dx beim Integral
\newcommand*{\mg}[1]{\mathbbm{#1}}      % Mengensymbol (statt \mathbb{#1}
\newcommand*{\NN}{\mg{N}}               %   natürliche Zahlen
\newcommand*{\PP}{\mg{P}}               %   Primzahlen
\newcommand*{\ZZ}{\mg{Z}}               %   ganze Zahlen
\newcommand*{\QQ}{\mg{Q}}               %   rationale Zahlen
\newcommand*{\RR}{\mg{R}}               %   reelle Zahlen
\newcommand*{\CC}{\mg{C}}               %   komplexe Zahlen
\newcommand*{\KK}{\mg{K}}               %   Körper
\newcommand*{\LL}{\mg{L}}               %   Körper oder Lösung
\newcommand*{\FF}{\mg{F}}               %   Field
\newcommand*{\DD}{\mg{D}}               %   Definitionsmenge
\newcommand*{\WW}{\mg{W}}               %   Wertebereich
\newcommand*{\EE}{\mg{E}}               %   Erwartungswert
\newcommand*{\BB}{\mg{B}}               %   Boolean
\newcommand*{\GG}{\mg{G}}               %   Gleitpunktzahlen

\newcommand*\arccot{\operatorname{arc cot}}
\newcommand*\arsinh{\operatorname{ar sinh}}
\newcommand*\arcosh{\operatorname{ar cosh}}
\newcommand*\artanh{\operatorname{ar tanh}}
\newcommand*\arcoth{\operatorname{ar coth}}
\newcommand*\Bild{\operatorname{Bild}}
\newcommand*\Kern{\operatorname{Kern}}
\newcommand*\Rg{\operatorname{Rg}}
\newcommand*\lnand{\operatorname{\overline{\wedge}}}
\newcommand*\lnor{\operatorname{\overline{\vee}}}
\newcommand*\lxor{\operatorname{\underline{\vee}}}
\newcommand*\llnor{\operatorname{\bar{\vee}}}
\newcommand*\llnand{\operatorname{\bar{\wedge}}}
\newcommand*\modulo{\operatorname{modulo}}
\newcommand*\boolAnd{\operatorname{\wedge}}
\newcommand*\boolOr{\operatorname{\vee}}
\newcommand*\sgn{\operatorname{sgn}}
\newcommand*\lb{\operatorname{lb}}
\newcommand*\Nst{\operatorname{Nst}}
\newcommand*\Eig{\operatorname{Eig}}
\newcommand*\rd{\operatorname{rd}}
\newcommand*\rot{\operatorname{rot}}

\newcommand*{\vektor}[1]{\boldsymbol{#1}}       %% Vektoren (fett oder mit Pfeil)
\newcommand*{\sdot}{\boldsymbol{\cdot}}  % Punkt für Skalarprodukt
\newcommand*{\Begriff}[1]{\textbf{#1}\index{#1}}

%% -----------------------------------------------------------------------------
%% Einheiten, für die Verwendung mit siunitx
\newcommand{\cm}{\centi\metre}

%% -----------------------------------------------------------------------------

%% -----------------------------------------------------------------------------
%% Gegen die meisten Overfull hboxes, aus
%% dctt, Axel Reichert, Message-ID: <a84us0$plqcm$7@ID-30533.news.dfncis.de>
\tolerance 1414
\hbadness 1414
\emergencystretch 1.5em
\hfuzz 0.3pt
\widowpenalty=10000
\vfuzz \hfuzz
\raggedbottom
%% -----------------------------------------------------------------------------
