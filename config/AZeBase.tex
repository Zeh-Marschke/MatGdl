%% =============================================================================
%% AZeBase
%% Grundlegende Festlegungen für LaTeX-Datei für Artikel, Report und
%% Buch
%% -----------------------------------------------------------------------------
%% AZeRahmen, Version 3.0.01 - 2025-01-25
%% =============================================================================

%% -- KOMA-spezifische Pakete einbinden ----------------------------------------
\usepackage{scrhack}          % beseitigt Inkompatibilitäten in KOMA-Script

\usepackage{typearea}         % Paket von KOMA-Script, ermöglicht Veränderung
                              %   vom Papierformat (Satzspiegelberechnung)
							  
\usepackage[T1]{fontenc}      % Kodierung der Schrift
\usepackage[utf8]{inputenc}   % Eingabezeichensatz

\usepackage[ngerman]{babel}   % Spracheinstellung

\usepackage{lmodern}          % verwendete Schriftart

\usepackage{graphicx}         % Einbinden von Grafiken

\usepackage{ifthen}           % Verwendung Boolescher Steuerung

\usepackage{tikz}             % Grafiken erstellen

\usepackage{xcolor}           % Farben

\usepackage[babel]{csquotes}  % automatische Anführungszeichen (Biber ?)
\usepackage{varioref}         % automatische Seitenangabe bei Querverweisen

\usepackage{mdwlist}          % engere Aufzählungen und Nummerierungen

\usepackage{lastpage}         % Nummer der letzten Seite

\usepackage{eurosym}          % Euro-Symbol

\usepackage[locale=DE]{siunitx}    
                              % Ausgabe von physikalischen Größen
							  
\usepackage[                  %% Hyperlinks
   breaklinks=true      % Links »überstehen« Zeilenumbruch 
     ,colorlinks        % Links erhalten Farben statt Kästen 
     ,linkcolor=black   % beeinflusst Inhaltsverzeichnis und Seitenzahlen 
     ,urlcolor=blue     % Farbe für URLs 
     ,citecolor=blue    % Farbe für Verweise auf Literatur 
     ,bookmarks         % Erzeugung von Bookmarks für PDF-Viewer 
     ,bookmarksnumbered % Nummerierung der Bookmarks 
]{hyperref} 

\usepackage[
  backend=biber               % Biber statt BibTeX verwenden
]{biblatex}

\usepackage{listings}         % Listings von Quellcode

% -- neue Packages, Bedeutung noch genauer ergründen und beschreiben
\usepackage{morewrites}       % ermöglicht mehr writes, 
                              % da ansonsten auf 16 beschränkt
\usepackage{subcaption}       % Tabellen und Grafiken nebeneinander
\usepackage{longtable}        % Tabellen über Seitengrenze
\usepackage{csvsimple}        % einlesen von Daten aus CSV-Datei
\usepackage{booktabs}         % Gestaltung Tabellen (toprule, midrule, ...)
\usepackage[figuresright]{rotating}
                              % Grafiken (auch Tabellen rotieren ???)
\usepackage{multirow}         % spaltenübergreifende Tabelleneinträge
\usepackage[makeindex, split, useindex]{splitidx}  
                              % Erstellen von Indizes
\newindex[Stichwortverzeichnis]{idx}
\ifthenelse{\equal{\INDEXNAMEN}{JA}}
  { \newindex[Namensverzeichnis]{names} }{ }  
\ifthenelse{\equal{\INDEXABKUERZUNGEN}{JA}}
  { \newindex[Abkürzungsverzeichnis]{abbr} }{ }  
\usepackage[font=it,labelfont=bf,format=plain]{caption}
                              % Bedeutung nicht ganz klar
                              % Formatierung von Bildunterschriften / Tabellenüberschriften ???
                              
\parindent = 0.0cm            % Einzug bei Absatzanfang

\addbibresource{\BIBLIOTHEK}

% Listings als Gleitobjekt definieren
% \usepackage{tocbasic} ist in scrartcl und scrbook (auch in scrreport?)
% enthalten
\DeclareNewTOC[
    type = listing
  , float
  , listname = {Liste der Listings}
]{listings}

%% =============================================================================
%% AZeMath
%% Grundlegende Festlegungen für LaTeX-Datei für Artikel, Report und
%% Buch für Mathematische Sachverhalte
%% -----------------------------------------------------------------------------
%% AZeRahmen, Version 3.0.01 - 2025-01-25
%% =============================================================================

%% -----------------------------------------------------------------------------
\usepackage{amssymb}          %% mathematische Symbole
\usepackage{amsmath}          %% Mathematikumgebung von AMS
\usepackage{amsfonts}         %% mathematische Schriftfamilien
\usepackage{framed}
\usepackage{bbm}              % mathematische Mengensymbole

\usepackage[amsmath,thmmarks,amsthm,framed]{ntheorem}

% --- Zerlegen in einzelnen Einheiten, wobei die Nummerierung vorne ist
\theoremstyle{change}
\theorembodyfont{\normalfont}
\newtheorem{Unit}{}[section]

%--- Definition von bunt gerahmten Theormen (ohne Nummern)
\theoremstyle{nonumberplain}
\def\theoremframecommand{\fcolorbox{green}{white}}
\newshadedtheorem{Definition}{Definition}
\def\theoremframecommand{\fcolorbox{red}{white}}
\newshadedtheorem{Theorem}{Theorem}
\newshadedtheorem{Satz}{Satz}
\def\theoremframecommand{\fcolorbox{blue}{white}}
\newshadedtheorem{Lemma}{Lemma}
\newshadedtheorem{Proposition}{Proposition}
\newshadedtheorem{Bemerkung}{Bemerkung}
\newshadedtheorem{Korollar}{Korollar}
\newshadedtheorem{Folgerung}{Folgerung}

\usepackage{gauss}            % Symbole für Gauss-Algorithmus
  \def\rowaddtolabel#1{\scriptstyle #1}  %% Faktoren bei Quelle und Ziel
\usepackage{polynom}          % Polynomdivision
%% -----------------------------------------------------------------------------

%% -----------------------------------------------------------------------------
%% mathematische Befehle 
%% -----------------------------------------------------------------------------
\newcommand*{\eoe}{\hfill $\Delta$}     % end of example
\newcommand*{\dd}[1]{\,\mathrm{d}#1}    % dx beim Integral
\newcommand*{\mg}[1]{\mathbbm{#1}}      % Mengensymbol (statt \mathbb{#1}
\newcommand*{\NN}{\mg{N}}               %   natürliche Zahlen
\newcommand*{\PP}{\mg{P}}               %   Primzahlen
\newcommand*{\ZZ}{\mg{Z}}               %   ganze Zahlen
\newcommand*{\QQ}{\mg{Q}}               %   rationale Zahlen
\newcommand*{\RR}{\mg{R}}               %   reelle Zahlen
\newcommand*{\CC}{\mg{C}}               %   komplexe Zahlen
\newcommand*{\KK}{\mg{K}}               %   Körper
\newcommand*{\LL}{\mg{L}}               %   Körper oder Lösung
\newcommand*{\FF}{\mg{F}}               %   Field
\newcommand*{\DD}{\mg{D}}               %   Definitionsmenge
\newcommand*{\WW}{\mg{W}}               %   Wertebereich
\newcommand*{\EE}{\mg{E}}               %   Erwartungswert
\newcommand*{\BB}{\mg{B}}               %   Boolean
\newcommand*{\GG}{\mg{G}}               %   Gleitpunktzahlen

\newcommand*\arccot{\operatorname{arc cot}}
\newcommand*\arsinh{\operatorname{ar sinh}}
\newcommand*\arcosh{\operatorname{ar cosh}}
\newcommand*\artanh{\operatorname{ar tanh}}
\newcommand*\arcoth{\operatorname{ar coth}}
\newcommand*\Bild{\operatorname{Bild}}
\newcommand*\Kern{\operatorname{Kern}}
\newcommand*\Rg{\operatorname{Rg}}
\newcommand*\lnand{\operatorname{\overline{\wedge}}}
\newcommand*\lnor{\operatorname{\overline{\vee}}}
\newcommand*\lxor{\operatorname{\underline{\vee}}}
\newcommand*\llnor{\operatorname{\bar{\vee}}}
\newcommand*\llnand{\operatorname{\bar{\wedge}}}
\newcommand*\modulo{\operatorname{modulo}}
\newcommand*\boolAnd{\operatorname{\wedge}}
\newcommand*\boolOr{\operatorname{\vee}}
\newcommand*\sgn{\operatorname{sgn}}
\newcommand*\lb{\operatorname{lb}}
\newcommand*\Nst{\operatorname{Nst}}
\newcommand*\Eig{\operatorname{Eig}}
\newcommand*\rd{\operatorname{rd}}
\newcommand*\rot{\operatorname{rot}}

\newcommand*{\vektor}[1]{\boldsymbol{#1}}       %% Vektoren (fett oder mit Pfeil)
\newcommand*{\sdot}{\boldsymbol{\cdot}}  % Punkt für Skalarprodukt
\newcommand*{\Begriff}[1]{\textbf{#1}\index{#1}}

%% -----------------------------------------------------------------------------
%% Einheiten, für die Verwendung mit siunitx
\newcommand{\cm}{\centi\metre}

%% -----------------------------------------------------------------------------

%% -----------------------------------------------------------------------------
%% Gegen die meisten Overfull hboxes, aus
%% dctt, Axel Reichert, Message-ID: <a84us0$plqcm$7@ID-30533.news.dfncis.de>
\tolerance 1414
\hbadness 1414
\emergencystretch 1.5em
\hfuzz 0.3pt
\widowpenalty=10000
\vfuzz \hfuzz
\raggedbottom
%% -----------------------------------------------------------------------------

%% =============================================================================
%% AZeMath
%% Grundlegende Festlegungen für LaTeX-Datei für Artikel, Report und
%% Buch für personalisierte Elemenete
%% - LogoA Andreas Zeh-Marschke: Quadrat, Kreis, Raute
%% - LogoB Andreas Zeh-Marschke: Namenskürzel AZe
%% - Adresse Andreas Zeh-Marschke
%% - Impressum 
%% - Titel (ohne Logo) Report, Buch
%% - DeckblattBuch (mit Logo, erste und zweite Seite)
%% - Titel Artikel
%% -----------------------------------------------------------------------------
%% AZeRahmen, Version 3.0.01 - 2025-01-25
%% =============================================================================

%% -----------------------------------------------------------------------------
%% Logo Andreas Zeh-Marschke: Quadrat, Kreis, Raute 
%% -----------------------------------------------------------------------------
\newcommand{\AZeLogoA}[2]{
  \begin{tikzpicture}[scale=#1]
    \draw[color=red, line width = #2 mm]
      (1,1) -- (1,-1) -- (-1,-1) -- (-1,1) -- (1,1);
    \draw[color=blue, line width = #2 mm]  
      (2,0) circle (1);
    \draw[color=green, line width = #2 mm] 
      (4,1) -- (5,0) -- (4,-1) -- (3,0) -- (4,1);
  \end{tikzpicture}
}
%% -----------------------------------------------------------------------------

%% -----------------------------------------------------------------------------
%% Logo Andreas Zeh-Marschke: Namenskürzel AZe 
%% -----------------------------------------------------------------------------
\newcommand{\AZeLogoB}[2]{
  \begin{tikzpicture}[scale=#1]
    \draw[line width=#2 mm, color=green] 
      (0.0,0.9) -- (0.3,1.6) -- (0.6,0.9);
    \draw[line width=#2 mm, color=red]   
      (0.0,1.2) -- (0.8,1.2) -- (0.0,0.0) -- (0.8,0.0); 
    \draw[line width=#2 mm, color=red]   
      (0.2,0.6) -- (1.0,0.6);
    \draw[line width=#2 mm, color=blue]  
      (0.6,0.3) -- (0.9,0.3);
    \draw[line width=#2 mm, color=blue]  
      (1.0,0.0) -- (0.6,0.0) -- (0.6,0.6) -- (1.0,0.6);
  \end{tikzpicture}
}
%% -----------------------------------------------------------------------------

%% -----------------------------------------------------------------------------
%% Adresse Andreas Zeh-Marschke 
%% -----------------------------------------------------------------------------
\newcommand{\AZeAdresse}{
  \begin{footnotesize} Dipl.-Mathematiker\end{footnotesize}
  Andreas Zeh-Marschke  
  \begin{footnotesize}M.Sc. Praktische Informatik \end{footnotesize}   \\
  Tauberring 16 b, 76344 Eggenstein-Leopoldshafen \\
  E-Mail Andreas(at)Zeh-Marschke.de \\
  Homepage http://www.Zeh-Marschke.de \\[2ex]
}
%% -----------------------------------------------------------------------------

%% -----------------------------------------------------------------------------
%% Impressum
%% -----------------------------------------------------------------------------
\newcommand{\Impressum}{
  \begin{tabular}{l l}
    \textbf{Impressum} & \\
    Copyright:         & \COPYRIGHT \\
    Version:           & \VERSION\ - \DATUM \\
    Layout und Satz:   & \AUTOR \\
  \end{tabular}
}
%% -----------------------------------------------------------------------------

%% -----------------------------------------------------------------------------
%% Titel 
%% -----------------------------------------------------------------------------
\newcommand{\schriftgroesse}[2]{\fontsize{#1}{#2}\selectfont}

%% --- Decklatt ohne Logo ------------------------------------------------------
\newcommand{\deckblattTitel}{
  \thispagestyle{empty}
  \begin{center} %{\ }
    \Huge{\TITEL}
    
    \Large{\UNTERTITEL}
    
    \Large{\AUTOR}
    
    \large{Version \VERSION}
    
  \end{center}
  \normalsize
}
%% -----------------------------------------------------------------------------

%% --- Deckblatt mit Logo ------------------------------------------------------
\newcommand{\deckblattBuch}{
  \deckblattTitel

  \begin{center}
  \AZeLogoA{1.0}{1.0}
  \end{center}
%% --- Seite 2 (Bibiographische Informationen)
  \newpage
  \AZeLogoB{1.0}{1.0} \\
  \AZeAdresse  \\[3ex]
  \Impressum
  
  \vspace*{5cm}
  Die Wiedergabe von Gebrauchsnamen, Handelsnamen, Warenbezeichnungen und so 
  weiter in diesem Werk berechtigt auch ohne besondere Kennzeichnung nicht zu 
  der Annahme, dass solche Namen im Sinne der Warenzeichen- und Markenschutz-
  Gesetzgebung als frei zu betrachten wären und daher von jedermann benutzt 
  werden dürfen.
  
  Dieses Werk ist von Andreas Zeh-Marschke unter der Lizenz
  Creative Commons Attribution 4.0 International (CC BY 4.0) freigegeben. 
  Eine Kopie der Lizenz findet sich unter 
  \url{https://creativecommons.org/licenses/by/4.0/}.

}
%% -----------------------------------------------------------------------------

%% -----------------------------------------------------------------------------
%% Titelei Artikel 
%% -----------------------------------------------------------------------------
\newcommand{\deckblattArtikel}{
  \thispagestyle{empty}
  \begin{center} {\ }
    \huge{\TITEL}
    
    \large{\UNTERTITEL}
    
    \large{\AUTOR}

    \DATUM 
    
    \VERSION
  \end{center}
}
%% -----------------------------------------------------------------------------
